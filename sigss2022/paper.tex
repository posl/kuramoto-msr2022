\documentclass[paper, technicalreport]{ieicej}
\usepackage{otf}
\usepackage[T1]{fontenc}
\usepackage{lmodern}
\usepackage{textcomp}
\usepackage{latexsym}
\usepackage[dvipdfmx]{graphicx}
\usepackage{color}
\usepackage{listings}


\lstset{
    language=Java,
    %commentstyle = {\color[cmyk]{1,0.4,1,0}}, %コメントのスタイル
    commentstyle = {}, %コメントのスタイル
    basicstyle={\ttfamily}, %書体の指定
    %basicstyle=\ttfamily\footnotesize,
    frame=tb, %フレームの指定
    framesep=3pt, %フレームと中身(コード)の間隔
    breaklines=true, %行が長くなった場合の改行
    %linewidth=7cm, %フレームの横幅
    lineskip=-0.4ex, %行間の調整
    tabsize=4, %Tabを何文字幅にするかの指定
    %numbers=left,
    caption={\ttfamily},
    keywordstyle=\color{blue},
    xrightmargin=0zw,
    escapechar=@
}

\renewcommand{\lstlistingname}{プログラム}

\setcounter{page}{1}

\field{}
\jtitle{GitHubにおけるバグ報告等の\\動画及び画像の活用実態に関する調査}
\etitle{The investigation of the usage of videos and images for bug reports in GitHub}

\authorlist{
    \alignorder=3
    \breakauthorline{3}
    \authorentry[kuramoto@posl.ait.kyushu-u.ac.jp]{蔵元 宏樹}{Hiroki Kuramto}{九州大学}
    \authorentry[ishimoto@posl.ait.kyushu-u.ac.jp]{石本 優太}{Yuta Ishimoto}{九州大学}
    \authorentry[shindo@posl.ait.kyushu-u.ac.jp]{新堂 風}{Kaze Shindo}{九州大学}
    \authorentry[kondo@ait.kyushu-u.ac.jp]{近藤 将成}{Masanari Kondo}{九州大学}
    \authorentry[kashiwa@ait.kyushu-u.ac.jp]{柏 祐太郎}{Yutaro Kashiwa}{九州大学}
    \authorentry[kamei@ait.kyushu-u.ac.jp]{亀井 靖高}{Yasutaka Kamei}{九州大学}
    \authorentry[ubayashi@ait.kyushu-u.ac.jp]{鵜林 尚靖}{Naoyasu Ubayashi}{九州大学}
}
\affiliate[九州大学]{九州大学}{Kyushu University}

\begin{document}

\begin{abstract}
    バグの症状の迅速で的確な把握は,デバッグ作業の効率に大きく影響する.
    バグの再現動画やバグの状況を表す画像は,デバッグ作業に有用であると期待できる.
    本研究の目的は,ソフトウェア開発現場におけるバグ報告動画及び画像の活用実態を明らかにすることである.
    GitHubで公開されている約4,000件のリポジトリを対象に,課題報告(Issueレポート)において動画及び画像の有無による影響を調査した.
    その結果として,動画及び画像が含まれる課題報告は,そうでないものに比べて,課題報告が解決するまでの時間が平均で,2.9\%\textasciitilde17.6\%増加し,
    またコメント数が平均で76.4\%\textasciitilde143\%増加した.
    一方で,最初のコメントがつくまでの時間が平均で20.1\%\textasciitilde25.4\%減少した.
    これらの結果の詳細について報告する.
\end{abstract}

\begin{keyword}
    GitHub,動画,画像,バグ報告,Issue
\end{keyword}

\begin{eabstract}
    A quick and accurate understanding of the symptoms of a bug can greatly affect the efficiency of debugging.
    Videos of bug reproductions and images showing the status of bugs are expected to be useful in the debugging process.
    The purpose of this study is to clarify the actual usage of bug report videos and images in software development sites.
    In this study, we investigated the impact of including video or images in the issue reports of about 4,000 public repositories in GitHub.
    As a result, the average time to resolve an issue report increased by 2.9\%\textasciitilde17.6\% 
    and the average number of comments increased by 76.4\%\textasciitilde143\% for issue reports that included video and images compared to those that did not.
    On the other hand, the average time until the first comment decreased by 20.1\%\textasciitilde25.4\%.
    We report the details of these results.
\end{eabstract}

\begin{ekeyword}
    GitHub, Movie, Image, Bug report, Issue
\end{ekeyword}


\maketitle

{\large
% はじめに
\section{はじめに\label{intro}}
ソフトウェア開発を行う際の重要な工程の一つとしてデバッグ作業があげられる.
デバッグ作業がソフトウェアの開発コストの50\%以上を占めるという結果が示されている\cite{Evaluating_Effectiveness},\cite{Auto_Repair}.
この問題を解決するためにバグ修正効率化の研究が盛んに行われている.

バグの症状の迅速で的確な把握は,バグの再現性を高め,デバッグ作業の効率に大きく影響する点で重要であると考えられる.
バグの再現動画及びバグの状況を表す画像は,バグの症状把握に有用であり,これを有効に活用することでデバッグ作業の効率化が期待できる.

本研究の目的は,ソフトウェア開発現場におけるバグ報告動画及び画像の活用実態を明らかにすることである.
GitHubで公開されている約4,000件のリポジトリを対象に,課題報告(Issueレポート)において動画及び画像の有無による影響を調査した.
その調査結果について,報告する.

本稿では,2節で本研究を行うに至った動機及び調査課題を述べ,
3節で本調査で用いる分析法について述べ,
4節で本研究で用いるデータセットについて述べる.
5節で調査内容及び調査結果を述べ,
6節で調査課題の回答と考察を述べる.
7節で妥当性への脅威について述べ,
最後に8節でまとめと今後の課題について述べる.

% 動機
\section{動機\label{motivation}}

GitHubでは,公開されたリポジトリに対して,そのオーナーだけでなく不特定多数のユーザーが開発に貢献しており,
不具合や意見の報告時に Issue機能が使用されている.
2021年以前は,Issue作成には文字入力の他に,GIF動画や画像のみ添付可能であり,
動画添付時には一度GIF動画に変換する手間があった.
現在では,MP4,MOVファイルを容易に添付できるようになり,今後動画の活用件数は増加すると考えられる.
しかし現時点では,動画及び画像の有無によるIssueへの影響について明らかにされていない.
本調査では,次の3つのRQを設定し,回答する.

\textbf{RQ1:動画及び画像の活用頻度はどの程度か.}\\
RQ1では,動画及び画像のどの程度の頻度で用いられているかを明らかにする.

\textbf{RQ2:動画及び画像の期待される効果は何か.}\\
RQ2では,動画及び画像の有無が,解決までの時間やコメントの回数などIssueのパラメータにどのように影響するかを明らかにする.

\textbf{RQ3:どのような報告内容に動画及び画像を用いるか.}\\
RQ3では,動画及び画像がどのような単語と併用されているかを調査することで,
どのような報告内容のときに動画及び画像が用いられているかを明らかにする.
% 本調査で用いる分析法
\section{本調査で用いる分析法\label{method}}
本節では,本調査で用いた検定法及び指標について紹介する.

\subsection{検定法}
本調査では,画像を含むIssue,動画を含むIssue,どちらも含まないIssueの3群の比較を行う.
差の検定法では一般に,$t$検定が用いられるが,3群のそれぞれの組み合わせに対して1回あたり有意水準$\alpha$で検定を3回繰り返すと、
真の有意水準が$\alpha$よりも大きくなってしまう.誤って有意差があると判断してしまうことにつながる(第一種の過誤).
3群以上の群間比較では,比較回数に応じて有意水準または有意確率を調節する多重比較が行われる.
本調査では,正規性及び等分散性を満たさないデータに対しても検定できる点において,本調査のデータの性質に合致するSteel-Dwass法を採用する.
Steel-Dwass法は,ウィルコクソンの順位和検定に基づくノンパラメトリックな多重比較法である.
ウィルコクソンの順位和検定は,以下の概念による検定法である.
データがいずれの群のものであるかに関わらず,全てのデータに対して順位を付ける.
さらに順位に重み付けした値を各群ごとに合計し,その順位和を比較する.
群間に差がなければ,それぞれの順位和は同程度になると考えられる.
これらの間に大きな差があるとすれば,ある群が順位の低いものになり,他の群が順位の高いものになる傾向があると判断する.
結果,各データ群は異なる母集団から抽出されたものである,あるいは,統計的に有意な差があると結論づける.
従って,帰無仮説は「すべての群の母集団分布は同じ」であり,棄却された時の解釈は「同じ母集団から採られたものではない」となる.
ここで注意しなければならない点は,「群Aは群Bよりも有意に大きい」と解釈できるのは,すべての群を通して等分散性が成立するときである.\cite{Interpretation_For_Nonparametric_Test}
また,順位和検定は順位を用いるため,平均値よりも中央値付近に強く影響を受ける.
その他,正規性の検定法には,コルモゴロフ・スミルノフ検定を採用し,等分散性の検定法には,Levene検定を採用する.
\\
\noindent{\bf{標本化について.}}
比較する各群から同一サイズの標本を抽出し,有意確率を算出する試行を10回行う.
10回の平均値を検定で得た有意確率とする.
サンプルサイズは、用いる有意水準,検出力,効果量及び用いる検定法によって決まる.
有意水準は一般的な0.05とする.その他の水準には,やや恣意的に検出力を0.80,効果量を0.10とする.
これらと,3群の対応のない比較の条件のもとで,各群から323サンプルとする.
サンプルサイズの算出には\textbf{G*Power}\cite{G*Power1}\cite{G*Power2}を利用した.
効果量は一般に先行研究からわかっている効果量を用いるが,本調査ではそれが得られなかったために, 0.10(効果量小),0.25(効果量中),0.40(効
果量大)の \textbf{Cohen(1988)}の基準から小程度の効果量を採用した.
なお,多重比較法における検出力の考え方についての枠組みは,サンプルサイズの設計という観点では,明確化されていない様である.\cite{Test_Power}
効果量,及び,検出力をやや恣意的に決定していることによるサンプルサイズへの影響は,6節で考察する.


\subsection{\textbf{tf\_idf}}
\textbf{tf\_idf}とは,文書中に含まれる単語の重要度を評価する指標である.
\textbf{tf}とはある語彙の出現頻度であり,\textbf{idf}とはある語彙の出現する文書数の逆数を取ったものである.
\textbf{tf\_idf}は\textbf{tf}と\textbf{idf}を掛け合わせたものである.語彙\textbf{t}が文書\textbf{d}に出現する回数を\textbf{f},
\textbf{d}の全単語数(重複を許す)を\textbf{T},\textbf{t}が出現する文書数を\textbf{n},
全文書数を\textbf{N}とすると,\textbf{d}内の\textbf{t}に対する\textbf{tf\_idf}は以下の式で求められる.
\begin{center}
  $\textbf{tf(t,d)=f/T}$ \\
  $\textbf{idf(t)=log(N/n)}$ \\
  $\textbf{tf\_idf(t,d)=tf(t,d)*idf(t)}$
\end{center}

\textbf{tf\_idf}値が大きい単語は,その文書において重要な意味を持つと考えられる.
Issue群ごとに,\textbf{tf\_idf}値が大きい単語を算出し,どんな内容のIssuenに対して動画及び画像が用いられる傾向があるかを分析するために用いる.

% データセット
\section{Dataset}
\label{sec:dataset}

In this section, we describe the data collection process and 
the overview of the collected dataset. 


\begin{figure*}[t]
\centering
% \includegraphics[width=1\linewidth]{./figures/data-category-trend.pdf}
\caption{ 
  An overview of the data collection
  }
\label{fig:data-collection-overview}
\end{figure*}


\subsection{Data Collection}
\fig{fig:data-collection-overview} shows
an overview of the data collection process.
We collected the data from the issues of 
the publicly available repositories on GitHub. 
We first select the repositories based on the following conditions:
\begin{itemize}
	\item the number of stars $\geq$ 10
	\item the number of issues $\geq$ 1
	\item the latest commit is in 2021
\end{itemize}
We used the number of stars for retrieving the repositories 
in which the owner may communicate with other developers and 
discard repositories developed by only the owner. 
In addition, we used the date of the latest commit 
because the major release of the feature is in 2021. 
Beforehand, the feature was the beta version. 
The number of the repositories that meet 
the conditions is 289,115. 
We randomly selected 4,173 repositories from them 
due to the execution time to collect issues. 
This number is less than 2\% of all repositories. 
However, the number of resolved issues 
in these repositories is already 770,656. 
We discuss the threat to validity of 
this process in \sec{sec:limitation}. 
This process was conducted from November 2021 to December 2021.


\begin{table}[t]
    \begin{center}
    \caption{The attributes we collected from the issues}
    \scalebox{0.85}[0.85]{
    \begin{tabular}{ll} 
        \toprule
        \multicolumn{1}{c}{\textbf{Attributes}} & \multicolumn{1}{c}{\textbf{Description}} \\ 
        \midrule
        $IssueResolvedTime$ & The time until the issue is resolved (day) \\
        $FirstCommentTime$ & The time until the first comment (day) \\
        $IssueCreatedYear$ & The year when the issue is created \\
        $\#comments$ & The number of comments \\
        $\#words$ & The number of words in the description \\
        $\#imgs$ & \# of attached images when the issue is created \\
        $\#vids$ & \# of attached videos when the issue is created \\
        $Words$ &  The words in the description \\
        \bottomrule
    \end{tabular}
    }
    \label{tab:issue-attr}
    \end{center}
\end{table}


We retrieved the attributes from the collected issues 
that are the data in the database. 
\tab{tab:issue-attr} shows the eight attributes 
retrieved from the issues. 

We used \texttt{PyGitHub}\masa{add citation} to execute GitHub API 
to retrieve the attributes from the issues. 
$IssueOpenTime$, $FirstCommentTime$,
and $IssueCreatedYear$ can be directly
retrieved by \texttt{PyGitHub}.
On the contrary,
$\#chars$, $\#imgs$, $\#movs$, and $Words$
need the conversion from the retrieved raw data.
We describe the details of the conversion. 
It should be noted that we converted the first description of 
issues into these four attribute values and 
ignore the comments and the title.
\masa{Is this description correct?}

The attached images and movies are transformed into 
URLs and put in the text of issues as the markdown format. 
The following URL is an example.

\begin{quote}
	https://user-images.githubusercontent.com/XXX.mp4
\end{quote}

\noindent{}
The part of XXX consists of alphanumeric, ``/'', and ``-''.
Hence, we prepared the regular expression for this URL and 
count the appearances of images and movies as $\#imgs$ and $\#movs$. 
The identification of images and movies is based on the extension of 
the URLs. 
Specifically, we used png, PNG, jpg, JPG, and jpeg as 
the extensions for images and 
gif, GIF, mp4, MP4, and mov as the extensions for movies.

We counted the number of characters as $\#chars$ and 
stored the words as $Words$ for all issues. 
In this process, if the description of the issue 
includes the URL, we exclude it from the description, 
and convert all issues into $\#chars$ and $Words$.
% For the issues that include the URL, we excluded them and 
% counted the number of words as $\#words$. 
% In addition, we counted the number of characters as $\#chars$. 

We noticed that some $IssueOpenTime$ values are less than zero. 
We decided that these values are invalid and the issues having 
these values are excluded from the dataset. 
In addition, to study the impact of movies and images on 
the communication in issues, 
we excluded issues that resolved less than 30 secs. 
This is because this issue resolution time is too short 
to communicate with each other. 
Specifically, we only retrieved the issues that meet 
the condition: $30\ sec \leq IssueOpenTime \leq 1\ year$.
The number of issues that meet this condition is 711,160 (92.23\%).


\begin{table}[h]
    \begin{center}
    \caption{The number of issues for each category}
    \begin{tabular}{llr}
        \toprule
         & \multicolumn{1}{c}{\textbf{Description}} & \multicolumn{1}{c}{\textbf{\#issues}} \\
        \midrule
        $Img$  & $\#imgs \geq 1$ & 33,079 (4.65\%)\\
        $Mov$  & $\#movs \geq 1$ & 3,819 (0.54\%)\\
        $None$ & Others & 674,793 (94.81\%)\\ 
        \bottomrule
    \end{tabular}
    \label{tab:issue-category}
    \end{center}
\end{table}


We classified the issues into three categories based on 
whether they have images and movies. 
\tab{tab:issue-category} shows the number of issues for each category. 
It should be noted that issues that have both 
the images and movies are counted for both 
$Img$ and $Mov$ categories. 

\begin{table*}[h]
    \begin{center}
    \caption{Examples of the retrieved issues with the values of the attributes}
    \begin{tabular}{c c c c c c c} 
      \toprule
      \textbf{IssueCreatedYear} &
      \textbf{ResolutionTime} &
      \textbf{Images} &
      \textbf{Videos} &
      \textbf{Comments} &
      \textbf{FirstCommentTime} &
      \textbf{DescriptionLength} \\
      \midrule
      2020 & 6.99861111 & 0 & 0 & 1 & 6.99861111 & 4430\\
      2020 & 41.9594329 & 1 & 0 & 3 & 17.7784722 & 85\\
      2020 & 43.8850579 & 0 & 0 & 2 & 0.49828704 & 56\\
      2020 & 44.0935532 & 0 & 0 & 4 & 0.91277778 & 33\\
      2020 & 0.14934028 & 0 & 0 & 8 & 0.08077546 & 244\\
      2020 & 59.5670949 & 2 & 0 & 5 & 0.39472222 & 102\\
      2020 & 74.9322569 & 0 & 0 & 0 & -          & 24\\
      \bottomrule
    \end{tabular}
    \label{tab:example-dataset}
    \end{center}
  \end{table*}


We extract a part of the retrieved issues in \tab{tab:example-dataset}. 
Each row corresponds to the values of the attributes of an issue. 

\begin{table}[t]
  \begin{center}
  \caption{The top-10 TFIDF words for each category}
  \begin{tabular}{l | c c c }
    \toprule
    & $Img$ & $Mov$ & $None$\\
    \midrule
    1 & image & when & dependabot\\
    2 & screenshot & dropdown & code\\
    3 & when & view & file\\
    4 & error & package & pullrequest\\
    5 & screen & issue & version\\
    6 & version & python & error\\
    7 & shot & height & use\\
    8 & file & react & lib\\
    9 & test & button & add\\
    10& code & text & commit\\
    \bottomrule
  \end{tabular}\\
  \label{tf-idf_result}
  \end{center}
\end{table}


We used $Words$ to compute the TFIDF values\masa{need citation} for each category. 
The TFIDF values will support researchers who investigate 
the differences in the appearances of words with and without 
images/movies. 
\tab{tag:tfidf} shows the top-10 TFIDF words in $Words$ 
for each category extracted from our dataset. 

\subsection{Dataset Description}


\begin{figure}[t]
\centering
\includegraphics[width=1\linewidth]{./figures/data-category-trend.pdf}
\caption{ 
  The proportions of issues for each category\masa{Is this correct?}
  }
\label{fig:data-cat-trend}
\end{figure}


\fig{fig:data-cat-trend} shows the proportions of 
issues in which developers attach either 
images or movies for each year. 
The y-axis shows the proportion. 
The proportions have slightly increased so far 
except for 2021. 
This may be because we collected the resolved issues 
in 2021. 
Given this result, both the images and movies are still 
not popular for developers. 


\begin{table*}[t]
  \begin{center}
  \caption{The statistics of the collected issues for each category}
  % \scalebox{0.85}[0.85]{
  \begin{tabular}{l c c c| c c c| c c c| c c c} 
    \toprule
    & \multicolumn{3}{c}{$IssueResolvedTime$} & \multicolumn{3}{c}{$FirstCommentTime$} & \multicolumn{3}{c}{$\#comments$} & \multicolumn{3}{c}{$\#chars$}\\
    & \textbf{$Img$} & \textbf{$Mov$} & \textbf{$None$} & \textbf{$Img$} & \textbf{$Mov$} & \textbf{$None$} & \textbf{$Img$} & \textbf{$Mov$} & \textbf{$None$} & \textbf{$Img$} & \textbf{$Mov$} & \textbf{$None$} \\ 
    \midrule
    Mean & 24.86 & 21.77 & 21.14 & 6.83 & 7.31 & 9.15 &  4.04 & 5.56 & 2.29 &  995.2 & 844.8 & 1350.6 \\
    Min & 30sec & 38sec & 30sec & 1sec & 1sec & 1sec &  0 & 0 & 0 &  6 & 1 & 0 \\
    1rd Q & 10.2hrs & 15.7hrs & 2.14hrs &  10.6min & 3.38min & 10.4min &  1 & 1 & 0 &  209 & 237 & 59 \\
    Mdn & 2.96 & 3.01 & 1.09 &  3.26hrs & 2.4hrs & 3.0hrs &  2 & 3 & 1 &  449 & 468 & 251 \\
    3rd Q & 17.30 & 13.12 & 10.80  & 1.38 & 1.18 & 1.84 &  5 & 7 & 3 &  901 & 897 & 880 \\
    Max & 364.56 & 361.83 & 364.99 &  364.10 & 337.61 & 364.61 &  283 & 120 & 439 &  $2.60E5$ & $5.89E4$ & $2.61E5$ \\
    S.D. & 55.64 & 53.96 & 53.64 &  28.27 & 32.77 & 34.39 &  6.873 & 8.710 & 4.619 &  $4.14E3$ & $2.23E3$ & $4.81E3$ \\
    \bottomrule
  \end{tabular}
  % }
  \label{issue_class_data}
  \end{center}
\end{table*}


\tab{tab:issue_stat_categories} shows the statistics
of the collected issues for the $Img$, $Mov$,
and $None$ categories.
The Mean row indicates the average values; 
the Min and Max rows indicate the minimum and maximum values; 
the 25th, 50th, and 75th rows indicate the percentile values; 
the S.D. row indicates the standard deviation values. 
We may not find the general conclusion; 
however, we observed differences across the categories. 
For example, the 25th, 50th, and 75th percentiles of 
the $Img$ and $Mov$ categories in $IssueResolvedTime$ are 
longer than those of the $None$ category. 

% 調査
\section{調査}
\subsection{動画・画像の活用件数の推移}
図\ref{content_trend}は,動画及び画像の活用件数の含有率の年別推移である.
2016年までは,画像の活用は0.01\%未満であり,動画に関しては0件であった.
2017年以降,活用件数は急激に増加し,2021年には,動画~0.72\%,画像~5.93\%であった.
2017年以降に急激に増加した要因に関しては,調査中である.

\begin{figure}[t]
  \begin{center}
      \includegraphics[scale=0.5]{./image/data_content_trends.pdf}
      \caption{動画・画像の含有率推移 \label{content_trend}}
  \end{center}
\end{figure}

\subsection{データの分析}

\begin{figure*}[t]
  \begin{center}
      \includegraphics[scale=1.1]{./image/dataset_plot.pdf}
      \caption{Issue群ごとの調査項目の分布及び代表値 \label{dataset_plot}}
  \end{center}
\end{figure*}

図\ref{dataset_plot}は,データの特性値を箱ひげ図及び表で表現したものである.
ただし,図\ref{dataset_plot}の箱図の最大値は,$3rd~Qu + 1.5 * IQR$以下の最大値である.
($IQR$:四分位範囲)
\\
% この様な処理は,データの中央値付近を見やすくする為にしばしば用いられる.
\noindent{\bf{正規性について.}}
%図\ref{dataset_plot}では,$Issue\_open\_time$,$First\_comment\_time$及び$Num\_of\_char$の平均値が,
%四分位範囲内に収まっていないことから正規分布から逸脱している.
%$Num\_of\_comments$に関しては,正規分布しているような印象を受ける.
%実際,$Img$の平均値 = 4.04,標準偏差$S.D.$ = 6.873であり,これから$\pm~3S.D.$の上限値は24.66と算出できる.
%これから,最大値283は外れ値であるとも考えられる.しかし,
$Kolmogorov-Smirnov$検定の結果,すべての項目に対して,正規性は有意水準0.05で棄却された.
\\
\noindent{\bf{等分散性について.}}
$Leneve$検定の結果を表\ref{levene_result}に示す.
帰無仮説は「すべてのIssue群間を通して等分散」であり,有意水準は0.20とする.

\begin{table}[h]
  \begin{center}
  \caption{等分散性検定・結果}
  \begin{tabular}{l|c c} 
    \hline
     & 有意確率 & 平均分散比 \\ 
    \hline \hline
    $Issue\_open\_time$ & 0.378 & 1.050 \\
    $First\_comment\_time$ & 0.296 & 1.301 \\
    $Num\_of\_comments$ & 0.001 & 2.459 \\
    $Num\_of\_char$ & 0.073 & 3.156 \\
    \hline
  \end{tabular}
  \label{levene_result}
  \end{center}
\end{table}

$Num\_of\_comments$及び$Num\_of\_char$は,等分散性が有意水準0.20で棄却される.
$Issue\_open\_time$及び$First\_comment\_time$は,有意確率が有意水準以上であることと,
平均分散比が~1~\textasciitilde ~1.5~程度であることから,等分散性を仮定する.


\subsection{統計的検定}
本小節の目的は,群間の分布の差が統計的に有意であるかの検定である.
検定には,正規性及び等分散性を仮定しない$Steel-Dwass$法を用いる.
$Steel-Dwass$法による検定結果を表\ref{Steel-Dwass_result}に示す.
ただし,有意水準は0.05とする.

\begin{table}[t]
  \begin{center}
  \caption{調査項目ごとの多重比較の結果}
  \begin{tabular}{l r|r}
    \hline
    調査項目 & 対 & 有意確率 \\
    \hline \hline
    $Issue\_open\_time$ & & \\
     & \bf{$Img$~vs~$None$} & **~0.002 \\
     & \bf{$Mov$~vs~$None$} & **~0.021 \\
     & \bf{$Img$~vs~$~Mov$} & 0.381 \\
    \hline
    $First\_comment\_time$ & & \\
     & \bf{$Img$~vs~$None$} & 0.764 \\
     & \bf{$Mov$~vs~$None$} & 0.351 \\
     & \bf{$Img$~vs~$~Mov$} & 0.404 \\
    \hline
    $Num\_of\_comments$ & & \\
     & \bf{$Img$~vs~$None$} & *~0.001 \\
     & \bf{$Mov$~vs~$None$} & *~0.001 \\
     & \bf{$Img$~vs~$~Mov$} & 0.211 \\
    \hline
    $Num\_of\_char$ & & \\
     & \bf{$Img$~vs~$None$} & *~0.001 \\
     & \bf{$Mov$~vs~$None$} & *~0.001 \\
     & \bf{$Img$~vs~$~Mov$} & 0.599 \\
    \hline
  \end{tabular}\\
  \small
    *~ : 両側検定で有意 ~~~ ** : 片側検定で有意\\
  \label{Steel-Dwass_result}
  \end{center}
\end{table}

$Img$及び$Mov$は,$None$に比べて,$Issue\_open\_time$が有意に大きく,
$Num\_of\_comments$及び$Num\_of\_char$については分布に何らかの有意な差があると解釈できる.
一方で,$First\_comment\_time$においては,有意差は認められなかった.
また,$Img$と$Mov$の比較では,すべての項目で有意差は認められなかった.
有意差が認められなかった項目については,有意差の有無は判断を保留する.

\subsection{出現単語分析}
Issueのテキストに含まれる単語を\textbf{tf\_idf}値に変換し,同じIssue群に属すもの同士をマージした結果,\textbf{tf\_idf}値が大きい順に10個の結果を表\ref{tf-idf_result}に示す.
"at"や"it","the"などの単語を除外する為,
名詞,動詞及び疑問詞のみを対象とした. \\

\begin{table}[t]
  \begin{center}
  \caption{Issue群ごとの\textbf{tf\_idf}上位10単語}
  \begin{tabular}{r | c c c }
     & $Img$ & $Mov$ & $None$\\
    \hline \hline
    1 & image & when & dependabot\\
    2 & screenshot & dropdown & code\\
    3 & when & view & file\\
    4 & error & package & pullrequest\\
    5 & screen & issue & version\\
    6 & version & python & error\\
    7 & shot & height & use\\
    8 & file & react & lib\\
    9 & test & button & add\\
    10& code & text & commit\\
  \end{tabular}\\
  \label{tf-idf_result}
  \end{center}
\end{table}
% 考察
\section{RQの回答と考察\label{research}}

本節では,5節の調査結果からRQ1~RQ3について回答し,サンプルサイズについて考察する.

5.1節の結果から,動画及び画像の活用頻度として以下の結果を得た.
\begin{itembox}{\textbf{RQ1:動画及び画像の活用頻度はどの程度か.}}
  ・2016年以前はほとんど用いられてこなかった.\\
  ・2017年に急激に増加し,以降増加傾向である.\\
  ・2021年には,動画0.72\%,画像5.93\%であった.
\end{itembox}

5.2節では,動画及び画像の有無で,Issueのパラメータがどのように変化するかを分析した.
また,5.3節では,各調査項目の群間の分布の差が統計的に有意であるかを確かめた.
動画及び画像を含むIssueは,そうでないIssueに比べて,
$Num\_of\_char$が平均で26.3\%\textasciitilde 37.5\%減少したことから,
動画及び画像を用いることでバグの症状を細かく記述せずに済むと推察される.
また,$First\_comment\_time$が平均で20.1\%\textasciitilde25.4\%減少し,
$Num\_of\_comments$が平均で76.4\%\textasciitilde143\%増加したことから,
動画及び画像のあるIssueは開発者の着手を早め,コメントを返信が得やすくなると推察される.
また,$Img$と$Mov$の比較検定では,すべての項目で有意差が認められなっかことから,
動画と画像の効果に有意な差はないと推察される.
%これらの結果から,動画及び画像の期待される効果として,以下が挙げられる.
\begin{itembox}{\textbf{RQ2:動画及び画像の期待される効果は何か.}}
  ・バグの症状を細かく記述せずに済む.\\
  ・開発者の着手を早め,コメントを得やすい.\\
  ・動画と画像の効果に有意な差はない.
\end{itembox}

5.4節では,出現単語分析を行うことで,どの様な報告内容の場合に動画及び画像を用いるかを確かめた.
$Img$では表示に関係する単語が多く,$Mov$では動きのある表示に関する単語が多く見られた.
一方で$None$では,プルリクエストやコミットなどリポジトリ管理に関する単語が多かった.

\begin{itembox}{\textbf{RQ3:どの様な報告に動画及び画像を用いるか.}}
  ・表示に関する問題に対して,画像を用いる.\\
  ・動きのある表示に関する問題に対して,動画を用いる.
\end{itembox}

一般にサンプルサイズは多ければ多いほど,母集団の特徴をより正確に抽出できる.
ところが検定においては,サンプルサイズが大きくなるにつれて,誤差に過敏になり有意確率が小さくなる傾向がある.
すなわち,検定で有意差を得たとしても,母集団分布の有意な差によるものか,サンプルサイズが大きすぎることによるものか判断できない.
従って,適切なサンプルサイズを前もって決めておくことが重要である.
本調査では,事前に効果量の目安が得られなかったために,やや恣意的に決定した.
検定結果を見ると,有意差がはっきりと認められた部分と,認められなかった部分が見られた$Issue\_open\_time$,$Num\_of\_comments$及び$Num\_of\_char$に対しては適切なサンプルサイズであったと推察する.
一方で$First\_comment\_time$に対しては,実際に有意と言える差が無いか,あるいは,サンプルサイズが小さすぎた可能性がある.

%また本調査では,すべての調査項目が正規分布に従わなかった.
%一般に,正規分布を仮定しない検定法よりも,正規分布を仮定する検定法は検出力に長けていると言われている.
%従って,データが正規分布に従わない場合,正規分布に近似するため適切な変数変換を行う.
%時系列データの変換には対数変換が多くの場合有効である.本調査の調査項目の一部は時系列データであり,
%変換が推奨される場面であるが,実際に適用させたところ平均値の群間順序関係が変化したことと,
%順位和検定により順位に変換されることから,対数変換を用いずに調査した.
% 妥当性への脅威
\section{妥当性への脅威\label{threats}}

%\noindent{\bf 内的妥当性.}

\noindent{\textbf{内的妥当性.}}本調査で用いた等分散性の検定の有意水準について,留意しなければならない点がある.
多重比較法は等分散性に関して,二群比較よりも鋭敏であり,有意水準0.20程度で行わなければ意味をなさないことが示されている.\cite{Test_Level}
さらには,有意水準0.50まで引き上げても,誤った結果を招く場合があると言われている.
その場合,$Issue\_open\_time$及び$First\_comment\_time$の等分散性は棄却され,
「動画及び画像があるIssueは,そうでないものに比べて$Issue\_open\_time$が有意に大きい」とした表現は,
「動画及び画像があるIssueは,そうでないものに比べて$Issue\_open\_time$に何らかの有意な差をもたらす」となる.

また,ボットが$None$に多くあらわれていることが,出現単語分析で明らかになった.
botを取り除く処理をしなければ,分布差が動画及び画像によるものとは断定できない.その点が考慮されていない.

Issueで報告される内容は,不具合に関するものだけではない.
事前調査では,新たな機能の提案に関する報告があることも確認した.
ただこの場合に関しても,動画及び画像により提案者の意図が伝わりやすくなることは,
バグの症状理解が早まることと本質的に同じと考えている.

\noindent{\textbf{外的妥当性.}}
本調査の調査対象は,条件に適合したリポジトリのみであり,すべての開発現場で同様の傾向があるとは限らない.
また,開発規模等のリポジトリの性質が考慮されていない.
% まとめと今後の展望
\section{Conclusion}
\label{sec:conclusion}

\masa{conclusion here}

Finally, we present future research directions with our dataset. 


\noindent
\textbf{The impact analysis of movies and images on software development.}
GitHub added the feature~\citep{github-video-blog} to easily 
share movies with GitHub to earn advantages 
in software development such as reproducing bugs easily in issues and 
explaining the background of changes in pull requests. 
However, no studies exist that investigate whether such movies 
impact software development. 
Hence, investigating the impact of movies on software development 
is a future research direction. 
Specifically, we study the reduction of issue resolution time, 
the response rate of developers, and 
the reduction of the number of comments 
in which developers write ``works for me''. 

\noindent
\textbf{Automated bug reproduction with image processing.}
Reproducing bugs is a time-consuming process 
in software development\masa{citation}.
Automating this process would support developers 
to quickly find and fix the cause of bugs. 
Hence, it is an important future research direction 
to automate bug reproduction. 
As the evolution of image processing with deep learning models, 
the accuracy of image processing significantly improves\masa{citation}. 
Hence, we apply such deep learning models to movies 
to implement automated bug reproduction tools. 
}

\section*{謝辞}
本研究の一部はXXXの助成を受けた.

%\bibliographystyle{jplain}
%\bibliography{references}

\bibliographystyle{junsrt}
\bibliography{references}
%\begin{thebibliography}{9}% 文献数が10未満の時 {9} 10以上の時は{99}
%    \bibitem{}
%\end{thebibliography}

\end{document}
