\section{結果と考察\label{research}}

画像及び動画を含むIssueは,そうでないIssueに比べて,
$Issue\_open\_time$が平均で,2.9\%\textasciitilde17.6\%増加し,
$Num\_of\_comments$が平均で76.4\%\textasciitilde143\%増加した.
一方で,$First\_comment\_time$が平均で20.1\%\textasciitilde25.4\%減少し,
$Num\_of\_char$が平均で26.3\%\textasciitilde 37.5\%減少した.

従って,動画及び画像を用いることには,次の様な効果があると推察できる.
\begin{itemize}
  \item 状況報告を細かく記述せずに済む
  \item 開発者の問題への着手を早める
  \item コメントが得やすくなる
\end{itemize}
また,$Issue\_open\_time$及び$Num\_of\_comments$が増加していることから,
動画及び画像は,比較的難しい問題に対して用いられているのではないかと推察される.

また,$Img$と$Mov$の間では,どの項目においても有意な差は認められなかったことから,
動画と画像のどちらを用いるかよりも,用いるか用いないかが重要であると推察される.

出現単語分析では,$Img$では表示に関係する単語が多く,
$Mov$では"dropdown","button"及び"react"など動きのある画面に関する単語が多い印象を受ける.
一方で,$None$では,コーディングに関する単語が多い結果であった.
$Mov$では"when"が1番目に,$Img$でも3番目にランクインした.これは,不具合の再現方法に関する記述である可能性が高い.
実際,目視調査で,不具合の再現動画が多いことを実験的に確かめた.
一方で,$None$では1番目に"dependabot"がランクインした.DependaBotは,最新のライブラリを推奨したり更新するボットであり,
設定により自動でIssueを生成することもある.
$Img$及び$Mov$と$None$の差は,ボットによる影響を受けている可能性がある.

単語分析の結果から,不具合が画面に関する場合に画像が用いられ,画面の動きに関する場合に動画が用いられ易いと推察できる.

一般にサンプルサイズは多ければ多いほど,母集団の特徴をより正確に抽出できる.
ところが検定においては,サンプルサイズが大きくなるにつれて,誤差に過剰に反応し,有意確率が小さくなる傾向がある.
すまわち,検定で有意差を得たとしても,母集団分布の有意な差によるものか,サンプルサイズが大きすぎることによるものか判断できないのである.
従って,適切なサンプルサイズを前もって決めておくことが重要である.
本調査では,事前に効果量の目安が得られなかったために,やや恣意的に決定した.
検定結果を見ると,有意差がはっきりと認められた部分と,認められなかった部分が見られた$Issue\_open\_time$,$Num\_of\_comments$及び$Num\_of\_char$に対しては適切なサンプルサイズであったと推察する.
一方で$First\_comment\_time$に対しては,実際に有意と言える差が無いか,あるいは,サンプルサイズが小さすぎた可能性がある.

また本調査では,すべての調査項目が正規分布に従わなかった.
一般に,正規分布を仮定しない検定法よりも,正規分布を仮定する検定法は検出力に長けていると言われている.
従って,データが正規分布に従わない場合,正規分布に近似するため適切な変数変換を行う.
時系列データの変換には対数変換が多くの場合有効である.本調査の調査項目の一部は時系列データであり,
変換が推奨される場面であるが,実際に適用させたところ平均値の群間順序関係が変化したことと,
順位和検定により順位に変換されることから,対数変換を用いずに調査した.