\section{はじめに\label{intro}}
ソフトウェア開発を行う際の重要な工程の一つとしてデバッグ作業があげられる.
デバッグ作業がソフトウェアの開発コストの50\%以上を占めるという結果が示されている\cite{Evaluating_Effectiveness},\cite{Auto_Repair}.
この問題を解決するためにバグ修正効率化の研究が盛んに行われている.

バグの症状の迅速で的確な把握は,バグの再現性を高め,デバッグ作業の効率に大きく影響する点で重要であると考えられる.
バグの再現動画及びバグの状況を表す画像は,バグの症状把握に有用であり,これを有効に活用することでデバッグ作業の効率化が期待できる.

本研究の目的は,ソフトウェア開発現場におけるバグ報告動画及び画像の活用実態を明らかにすることである.
GitHubで公開されている約4,000件のリポジトリを対象に,課題報告(Issueレポート)において動画及び画像の有無による影響を調査した.
その調査結果について,報告する.

本稿では,2節で本研究を行うに至った動機を述べ,
3節で本調査で用いる分析法について述べ,
4節で本研究で用いるデータセットについて述べる.
5節で調査内容と検定結果を述べ,
6節で調査結果と考察を述べる.
7節で妥当性への脅威について述べ,
最後に8節でまとめと今後の課題について述べる.
