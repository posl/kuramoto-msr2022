\section{本調査で用いる分析法\label{method}}
本節では,本調査で用いた検定法及び指標について紹介する.

\subsection{検定法}
本調査では,画像を含むIssue,動画を含むIssue,どちらも含まないIssueの3群の比較を行う.
差の検定法では一般に,$t$検定が用いられるが,3群のそれぞれの組み合わせに対して1回あたり有意水準$\alpha$で検定を3回繰り返すと、
真の有意水準が$\alpha$よりも大きくなってしまう.誤って有意差があると判断してしまうことにつながる(第一種の過誤).
3群以上の群間比較では,比較回数に応じて有意水準または有意確率を調節する多重比較が行われる.
本調査では,正規性及び等分散性を満たさないデータに対しても検定できる点において,本調査のデータの性質に合致するSteel-Dwass法を採用する.
Steel-Dwass法は,ウィルコクソンの順位和検定に基づくノンパラメトリックな多重比較法である.
ウィルコクソンの順位和検定は,以下の概念による検定法である.
データがいずれの群のものであるかに関わらず,全てのデータに対して順位を付ける.
さらに順位に重み付けした値を各群ごとに合計し,その順位和を比較する.
群間に差がなければ,それぞれの順位和は同程度になると考えられる.
これらの間に大きな差があるとすれば,ある群が順位の低いものになり,他の群が順位の高いものになる傾向があると判断する.
結果,各データ群は異なる母集団から抽出されたものである,あるいは,統計的に有意な差があると結論づける.
従って,帰無仮説は「すべての群の母集団分布は同じ」であり,棄却された時の解釈は「同じ母集団から採られたものではない」となる.
ここで注意しなければならない点は,「群Aは群Bよりも有意に大きい」と解釈できるのは,すべての群を通して等分散性が成立するときである.\cite{Interpretation_For_Nonparametric_Test}
また,順位和検定は順位を用いるため,平均値よりも中央値付近に強く影響を受ける.
その他,正規性の検定法には,コルモゴロフ・スミルノフ検定を採用し,等分散性の検定法には,Levene検定を採用する.
\\
\noindent{\bf{標本化について.}}
比較する各群から同一サイズの標本を抽出し,有意確率を算出する試行を10回行う.
10回の平均値を検定で得た有意確率とする.
サンプルサイズは、用いる有意水準,検出力,効果量及び用いる検定法によって決まる.
有意水準は一般的な0.05とする.その他の水準には,やや恣意的に検出力を0.80,効果量を0.10とする.
これらと,3群の対応のない比較の条件のもとで,各群から323サンプルとする.
サンプルサイズの算出には\textbf{G*Power}\cite{G*Power1}\cite{G*Power2}を利用した.
効果量は一般に先行研究からわかっている効果量を用いるが,本調査ではそれが得られなかったために, 0.10(効果量小),0.25(効果量中),0.40(効
果量大)の \textbf{Cohen(1988)}の基準から小程度の効果量を採用した.
なお,多重比較法における検出力の考え方についての枠組みは,サンプルサイズの設計という観点では,明確化されていない様である.\cite{Test_Power}
効果量,及び,検出力をやや恣意的に決定していることによるサンプルサイズへの影響は,6節で考察する.


\subsection{\textbf{tf\_idf}}
\textbf{tf\_idf}とは,文書中に含まれる単語の重要度を評価する指標である.
\textbf{tf}とはある語彙の出現頻度であり,\textbf{idf}とはある語彙の出現する文書数の逆数を取ったものである.
\textbf{tf\_idf}は\textbf{tf}と\textbf{idf}を掛け合わせたものである.語彙\textbf{t}が文書\textbf{d}に出現する回数を\textbf{f},
\textbf{d}の全単語数(重複を許す)を\textbf{T},\textbf{t}が出現する文書数を\textbf{n},
全文書数を\textbf{N}とすると,\textbf{d}内の\textbf{t}に対する\textbf{tf\_idf}は以下の式で求められる.
\begin{center}
  $\textbf{tf(t,d)=f/T}$ \\
  $\textbf{idf(t)=log(N/n)}$ \\
  $\textbf{tf\_idf(t,d)=tf(t,d)*idf(t)}$
\end{center}

\textbf{tf\_idf}値が大きい単語は,その文書において重要な意味を持つと考えられる.
Issue群ごとに,\textbf{tf\_idf}値が大きい単語を算出し,どんな内容のIssuenに対して動画及び画像が用いられる傾向があるかを分析するために用いる.
