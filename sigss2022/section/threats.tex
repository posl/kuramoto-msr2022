\section{妥当性への脅威\label{threats}}

%\noindent{\bf 内的妥当性.}

\noindent{\textbf{内的妥当性.}}本調査で用いた等分散性の検定の有意水準について,留意しなければならない点がある.
多重比較法は等分散性に関して,二群比較よりも鋭敏であり,有意水準0.20程度で行わなければ意味をなさないことが示されている.\cite{Test_Level}
さらには,有意水準0.50まで引き上げても,誤った結果を招く場合があると言われている.
その場合,$Issue\_open\_time$及び$First\_comment\_time$の等分散性は棄却され,
「動画及び画像があるIssueは,そうでないものに比べて$Issue\_open\_time$が有意に大きい」とした表現は,
「動画及び画像があるIssueは,そうでないものに比べて$Issue\_open\_time$に何らかの有意な差をもたらす」となる.

また,ボットが$None$に多くあらわれていることが,出現単語分析で明らかになった.
botを取り除く処理をしなければ,分布差が動画及び画像によるものとは断定できない.その点が考慮されていない.

Issueで報告される内容は,不具合に関するものだけではない.
事前調査では,新たな機能の提案に関する報告があることも確認した.
ただこの場合に関しても,動画及び画像により提案者の意図が伝わりやすくなることは,
バグの症状理解が早まることと本質的に同じと考えている.

\noindent{\textbf{外的妥当性.}}
本調査の調査対象は,条件に適合したリポジトリのみであり,すべての開発現場で同様の傾向があるとは限らない.
また,開発規模等のリポジトリの性質が考慮されていない.