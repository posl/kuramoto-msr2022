% \documentclass[sigconf,review]{acmart}
\documentclass[sigconf]{acmart}
\acmConference[ICPC 2022]{ICPC '22: Proceedings of the 30th IEEE/ACM International Conference on Program Comprehension}{May 16–17, 2022}{Pittsburgh, PA, USA}

% ------------- Packages Used --------------------------------------------------
% They help us to produce a better looking document ;-)
% ------------------------------------------------------------------------------
%\usepackage[tight,footnotesize]{subfigure}

\usepackage{bm}
\usepackage{tabularx}
\usepackage{graphicx}
\usepackage{framed}
\usepackage{colortbl}
\usepackage{balance}
\usepackage{multirow}
\usepackage{booktabs}
\usepackage{varwidth}
\usepackage{collectbox}
\usepackage{comment}
\usepackage{url}
\usepackage{txfonts}
\usepackage{cite}
\usepackage{lscape}
\usepackage{longtable}
\usepackage{caption}
\usepackage{adjustbox}
\usepackage{afterpage}
\usepackage{lipsum}
\usepackage{graphics}
\usepackage{xcolor}

\usepackage{tcolorbox}
\tcbuselibrary{raster,skins}
\definecolor{light-gray}{gray}{0.85}

\usepackage{float}
\usepackage{listings}
\lstset{frame=tb,
  language=Java,
  aboveskip=2mm,
  belowskip=2mm,
  showstringspaces=false,
  columns=flexible,
  basicstyle={\small\ttfamily},
  numbers=none,
  numberstyle=\tiny\color{red},
  keywordstyle=\color{blue},
  commentstyle=\color{red},
  stringstyle=\color{red},
  breaklines=true,
  breakatwhitespace=true,
  tabsize=1
}

%   ACM Style
%\usepackage{lcsect}
% ------------------------------------------------------------------------------











% ---------- Special commands for annotating the paper's text ------------------
\let\mymarginpar\marginparm
\marginparwidth=1cm
\marginparsep=5pt
\def\fig#1{Figure~\ref{#1}}
\def\tab#1{Table~\ref{#1}}
\def\eqn#1{Equation~\ref{#1}}
\def\sec#1{Section~\ref{#1}}

\usepackage{xspace}
\def\et{et\ al.\xspace}
\def\ie{i.e.\xspace}
\def\eg{e.g.\xspace}
\def\wrt{w.\,r.\,t.\xspace}
\def\aka{a.k.a.\xspace}



\newcommand{\masa}[1]{\textcolor{blue}{{\it [Masa says: #1]}}}



% ------------------------- SYMBOLS OF SELF NAMES OFTEN USED -------------------


% \usepackage[colorlinks,bookmarksopen,bookmarksnumbered,citecolor=red,urlcolor=red]{hyperref}
%\smartqed  % flush right qed marks, e.g. at end of proof
%\smartqed  % flush right qed marks, e.g. at end of proof
%

\usepackage[tight,footnotesize]{subfigure}
\usepackage{cite}
\usepackage{graphicx}
% \usepackage[numbers,sort&compress]{natbib}
% \usepackage{natbib}

\usepackage{comment}


\begin{document}


% \title{A Dataset of Issues with Movies and Images in GitHub}
\title{Do Visual Issue Reports Help Developers Fix Bugs?}
\subtitle{-- A Preliminary Study of Using Videos and Images to Report Issues on GitHub --}

% \author{Hiroki Kuramoto, Yuta Ishimoto, Kaze Shindo, Masanari Kondo, Yutaro Kashiwa, Yasutaka Kamei, Naoyasu Ubayashi}
% \affiliation{
%   \institution{Principles of Software engineering and programming Languages Lab. (POSL), Kyushu University}
% 	\country{Japan}
% }
% \email{{kuramoto,ishimoto,shindo}@posl.ait.kyushu-u.ac.jp,{kondo,kashiwa,kamei,ubayashi}@ait.kyushu-u.ac.jp}

% ============================================================
% Research questions
% ============================================================
% \newcommand{\RQone}{Is using movies and images in issues popular in GitHub?}
% \newcommand{\RQtwo}{Do the issues with and without visualization have any differences?}
\newcommand{\RQone}{Do the visual issue reports reduce effort to report bugs?}
\newcommand{\RQtwo}{Do the visual issue reports receive more/faster responses than non-visual issue reports?}
\newcommand{\RQthree}{Do the visual issue reports get resolved faster than non-visual issue reports?}
% ============================================================

%\begin{tcolorbox}[enhanced,title=#1, attach boxed title to top left= {xshift=0mm, yshift*=-\tcboxedtitleheight/2}]
\def\summarybox#1{
%\medskip
\begin{tcolorbox}
    #1
\end{tcolorbox}
}


%http://dl.acm.org/ccs.cfm
%TODO: This should be changed
\begin{CCSXML}
<ccs2012>
<concept>
<concept_id>10011007.10011074.10011111.10011696</concept_id>
<concept_desc>Software and its engineering~Maintaining software</concept_desc>
<concept_significance>500</concept_significance>
</concept>
</ccs2012>
\end{CCSXML}

\ccsdesc[500]{Software and its engineering~Maintaining software}

\begin{abstract}
In distributed development, asynchronous communication 
across developers is an essential part. 
To promote asynchronous communication across developers, 
GitHub has recently released a new feature to share movies on GitHub. 
This feature has the potential to make asynchronous communication 
across developers quickly. 
However, no database or studies about this topic exist so far. 
In this paper, we present the first database that 
consists of movies and images in publicly available repositories on GitHub. 
Specifically, we collected \masa{number} videos and \masa{number} images 
from \masa{number} issues on \masa{number} publicly available repositories.
\end{abstract}
\keywords{GitHub, Issues, Videos, Images}

\author[]{Anonymous\\~}
\maketitle

\section{Introduction}
\label{sec:intro}
The question "What makes a good issue report?'' has been studied for decades and is still the ultimate research question for many studies aiming to improve the quality of issue reports~\citep{DBLP:conf/icse/HerzigJZ13}\citep{zimmermann2010TSE}\citep{DBLP:conf/eclipse/BettenburgJSWPZ07}. Issue reports (a.k.a. bug reports) often lack the information necessary for developers to reproduce bugs~\citep{DBLP:conf/msr/JoorabchiMM14}\citep{DearGitHub}. 
For example, Zimmermann~\et~\citep{zimmermann2010TSE} report that stack traces and steps for reproducing a bug are considered to be helpful by developers. But, it is difficult for users to provide this information, and it is often missing or incorrect. 
This mismatch between what developers need and what reporters can provide can often delay the fixing of bugs~\citep{DBLP:conf/msr/JoorabchiMM14}. In addition, many studies have reported that the quality of issue reports impacts both the issue resolution time~\citep{DBLP:conf/cscw/BreuPSZ10}\citep{DBLP:conf/icse/GuoZNM10} and the issue resolution rate~\citep{DBLP:conf/compsac/ZouXZCL15}\citep{DBLP:conf/icse/ZimmermannNGM12}. 

To facilitate developers' bug-reproduction work, GitHub launched a new feature that allows users to share videos (e.g., mp4 files) in May 2021~\citep{github-video-blog}. Using such videos, reports can be made to developers about the details of bugs by recording the symptoms, reproduction steps, and other important aspects of a comprehensive bug report. These visual images can help developers understand the nature of the bug, and what users were doing when the bug occurred. While such visual issue reports have the potential to significantly improve the bug-fixing process, no studies have empirically examined this impact. 

In this paper, we conduct a preliminary study to identify the characteristics of visual issue reports by comparing them with non-visual issue reports.  In addition, we provide the dataset used in this study on \kashiwa{Where?}\footnote{URL here}, to promote future studies using visual issue reports. This dataset consists of videos and images in publicly available repositories on GitHub. Specifically, we collected 1,230 videos and 18,760 images from 226,286 issue reports on 4,173 publicly available repositories.


%Our initial analysis reveals that (i) visual issue reports still require reporters to write similar amounts of texts to describe bugs; (ii) visual issue reports are 3-times likely to receive more comments than reports without images or videos; and (iii) resolution time of visual issue reports is longer than that of other issue reports. 
Our initial analysis reveals that 
(i) issue reports with images are described in fewer words than non-visual issues; 
(ii) visual issue reports do not lead to active discussions in the number of words and the first response time; and 
(iii) resolution times of visual issue reports are not significantly changed compared to other issue reports. 
%The main contribution of this paper is opening a new research perspective to study visual issue reports and preparing a new dataset for this research. 
\section{Dataset}
\label{sec:dataset}

In this section, we describe the data collection process and 
the overview of the collected dataset. 


\begin{figure*}[t]
\centering
% \includegraphics[width=1\linewidth]{./figures/data-category-trend.pdf}
\caption{ 
  An overview of the data collection
  }
\label{fig:data-collection-overview}
\end{figure*}


\subsection{Data Collection}
\fig{fig:data-collection-overview} shows
an overview of the data collection process.
We collected the data from the issues of 
the publicly available repositories on GitHub. 
We first select the repositories based on the following conditions:
\begin{itemize}
	\item the number of stars $\geq$ 10
	\item the number of issues $\geq$ 1
	\item the latest commit is in 2021
\end{itemize}
We used the number of stars for retrieving the repositories 
in which the owner may communicate with other developers and 
discard repositories developed by only the owner. 
In addition, we used the date of the latest commit 
because the major release of the feature is in 2021. 
Beforehand, the feature was the beta version. 
The number of the repositories that meet 
the conditions is 289,115. 
We randomly selected 4,173 repositories from them 
due to the execution time to collect issues. 
This number is less than 2\% of all repositories. 
However, the number of resolved issues 
in these repositories is already 770,656. 
We discuss the threat to validity of 
this process in \sec{sec:limitation}. 
This process was conducted from November 2021 to December 2021.


\begin{table}[t]
    \begin{center}
    \caption{The attributes we collected from the issues}
    \scalebox{0.85}[0.85]{
    \begin{tabular}{ll} 
        \toprule
        \multicolumn{1}{c}{\textbf{Attributes}} & \multicolumn{1}{c}{\textbf{Description}} \\ 
        \midrule
        $IssueResolvedTime$ & The time until the issue is resolved (day) \\
        $FirstCommentTime$ & The time until the first comment (day) \\
        $IssueCreatedYear$ & The year when the issue is created \\
        $\#comments$ & The number of comments \\
        $\#words$ & The number of words in the description \\
        $\#imgs$ & \# of attached images when the issue is created \\
        $\#vids$ & \# of attached videos when the issue is created \\
        $Words$ &  The words in the description \\
        \bottomrule
    \end{tabular}
    }
    \label{tab:issue-attr}
    \end{center}
\end{table}


We retrieved the attributes from the collected issues 
that are the data in the database. 
\tab{tab:issue-attr} shows the eight attributes 
retrieved from the issues. 

We used \texttt{PyGitHub}\masa{add citation} to execute GitHub API 
to retrieve the attributes from the issues. 
$IssueOpenTime$, $FirstCommentTime$,
and $IssueCreatedYear$ can be directly
retrieved by \texttt{PyGitHub}.
On the contrary,
$\#chars$, $\#imgs$, $\#movs$, and $Words$
need the conversion from the retrieved raw data.
We describe the details of the conversion. 
It should be noted that we converted the first description of 
issues into these four attribute values and 
ignore the comments and the title.
\masa{Is this description correct?}

The attached images and movies are transformed into 
URLs and put in the text of issues as the markdown format. 
The following URL is an example.

\begin{quote}
	https://user-images.githubusercontent.com/XXX.mp4
\end{quote}

\noindent{}
The part of XXX consists of alphanumeric, ``/'', and ``-''.
Hence, we prepared the regular expression for this URL and 
count the appearances of images and movies as $\#imgs$ and $\#movs$. 
The identification of images and movies is based on the extension of 
the URLs. 
Specifically, we used png, PNG, jpg, JPG, and jpeg as 
the extensions for images and 
gif, GIF, mp4, MP4, and mov as the extensions for movies.

We counted the number of characters as $\#chars$ and 
stored the words as $Words$ for all issues. 
In this process, if the description of the issue 
includes the URL, we exclude it from the description, 
and convert all issues into $\#chars$ and $Words$.
% For the issues that include the URL, we excluded them and 
% counted the number of words as $\#words$. 
% In addition, we counted the number of characters as $\#chars$. 

We noticed that some $IssueOpenTime$ values are less than zero. 
We decided that these values are invalid and the issues having 
these values are excluded from the dataset. 
In addition, to study the impact of movies and images on 
the communication in issues, 
we excluded issues that resolved less than 30 secs. 
This is because this issue resolution time is too short 
to communicate with each other. 
Specifically, we only retrieved the issues that meet 
the condition: $30\ sec \leq IssueOpenTime \leq 1\ year$.
The number of issues that meet this condition is 711,160 (92.23\%).


\begin{table}[h]
    \begin{center}
    \caption{The number of issues for each category}
    \begin{tabular}{llr}
        \toprule
         & \multicolumn{1}{c}{\textbf{Description}} & \multicolumn{1}{c}{\textbf{\#issues}} \\
        \midrule
        $Img$  & $\#imgs \geq 1$ & 33,079 (4.65\%)\\
        $Mov$  & $\#movs \geq 1$ & 3,819 (0.54\%)\\
        $None$ & Others & 674,793 (94.81\%)\\ 
        \bottomrule
    \end{tabular}
    \label{tab:issue-category}
    \end{center}
\end{table}


We classified the issues into three categories based on 
whether they have images and movies. 
\tab{tab:issue-category} shows the number of issues for each category. 
It should be noted that issues that have both 
the images and movies are counted for both 
$Img$ and $Mov$ categories. 

\begin{table*}[h]
    \begin{center}
    \caption{Examples of the retrieved issues with the values of the attributes}
    \begin{tabular}{c c c c c c c} 
      \toprule
      \textbf{IssueCreatedYear} &
      \textbf{ResolutionTime} &
      \textbf{Images} &
      \textbf{Videos} &
      \textbf{Comments} &
      \textbf{FirstCommentTime} &
      \textbf{DescriptionLength} \\
      \midrule
      2020 & 6.99861111 & 0 & 0 & 1 & 6.99861111 & 4430\\
      2020 & 41.9594329 & 1 & 0 & 3 & 17.7784722 & 85\\
      2020 & 43.8850579 & 0 & 0 & 2 & 0.49828704 & 56\\
      2020 & 44.0935532 & 0 & 0 & 4 & 0.91277778 & 33\\
      2020 & 0.14934028 & 0 & 0 & 8 & 0.08077546 & 244\\
      2020 & 59.5670949 & 2 & 0 & 5 & 0.39472222 & 102\\
      2020 & 74.9322569 & 0 & 0 & 0 & -          & 24\\
      \bottomrule
    \end{tabular}
    \label{tab:example-dataset}
    \end{center}
  \end{table*}


We extract a part of the retrieved issues in \tab{tab:example-dataset}. 
Each row corresponds to the values of the attributes of an issue. 

\begin{table}[t]
  \begin{center}
  \caption{The top-10 TFIDF words for each category}
  \begin{tabular}{l | c c c }
    \toprule
    & $Img$ & $Mov$ & $None$\\
    \midrule
    1 & image & when & dependabot\\
    2 & screenshot & dropdown & code\\
    3 & when & view & file\\
    4 & error & package & pullrequest\\
    5 & screen & issue & version\\
    6 & version & python & error\\
    7 & shot & height & use\\
    8 & file & react & lib\\
    9 & test & button & add\\
    10& code & text & commit\\
    \bottomrule
  \end{tabular}\\
  \label{tf-idf_result}
  \end{center}
\end{table}


We used $Words$ to compute the TFIDF values\masa{need citation} for each category. 
The TFIDF values will support researchers who investigate 
the differences in the appearances of words with and without 
images/movies. 
\tab{tag:tfidf} shows the top-10 TFIDF words in $Words$ 
for each category extracted from our dataset. 

\subsection{Dataset Description}


\begin{figure}[t]
\centering
\includegraphics[width=1\linewidth]{./figures/data-category-trend.pdf}
\caption{ 
  The proportions of issues for each category\masa{Is this correct?}
  }
\label{fig:data-cat-trend}
\end{figure}


\fig{fig:data-cat-trend} shows the proportions of 
issues in which developers attach either 
images or movies for each year. 
The y-axis shows the proportion. 
The proportions have slightly increased so far 
except for 2021. 
This may be because we collected the resolved issues 
in 2021. 
Given this result, both the images and movies are still 
not popular for developers. 


\begin{table*}[t]
  \begin{center}
  \caption{The statistics of the collected issues for each category}
  % \scalebox{0.85}[0.85]{
  \begin{tabular}{l c c c| c c c| c c c| c c c} 
    \toprule
    & \multicolumn{3}{c}{$IssueResolvedTime$} & \multicolumn{3}{c}{$FirstCommentTime$} & \multicolumn{3}{c}{$\#comments$} & \multicolumn{3}{c}{$\#chars$}\\
    & \textbf{$Img$} & \textbf{$Mov$} & \textbf{$None$} & \textbf{$Img$} & \textbf{$Mov$} & \textbf{$None$} & \textbf{$Img$} & \textbf{$Mov$} & \textbf{$None$} & \textbf{$Img$} & \textbf{$Mov$} & \textbf{$None$} \\ 
    \midrule
    Mean & 24.86 & 21.77 & 21.14 & 6.83 & 7.31 & 9.15 &  4.04 & 5.56 & 2.29 &  995.2 & 844.8 & 1350.6 \\
    Min & 30sec & 38sec & 30sec & 1sec & 1sec & 1sec &  0 & 0 & 0 &  6 & 1 & 0 \\
    1rd Q & 10.2hrs & 15.7hrs & 2.14hrs &  10.6min & 3.38min & 10.4min &  1 & 1 & 0 &  209 & 237 & 59 \\
    Mdn & 2.96 & 3.01 & 1.09 &  3.26hrs & 2.4hrs & 3.0hrs &  2 & 3 & 1 &  449 & 468 & 251 \\
    3rd Q & 17.30 & 13.12 & 10.80  & 1.38 & 1.18 & 1.84 &  5 & 7 & 3 &  901 & 897 & 880 \\
    Max & 364.56 & 361.83 & 364.99 &  364.10 & 337.61 & 364.61 &  283 & 120 & 439 &  $2.60E5$ & $5.89E4$ & $2.61E5$ \\
    S.D. & 55.64 & 53.96 & 53.64 &  28.27 & 32.77 & 34.39 &  6.873 & 8.710 & 4.619 &  $4.14E3$ & $2.23E3$ & $4.81E3$ \\
    \bottomrule
  \end{tabular}
  % }
  \label{issue_class_data}
  \end{center}
\end{table*}


\tab{tab:issue_stat_categories} shows the statistics
of the collected issues for the $Img$, $Mov$,
and $None$ categories.
The Mean row indicates the average values; 
the Min and Max rows indicate the minimum and maximum values; 
the 25th, 50th, and 75th rows indicate the percentile values; 
the S.D. row indicates the standard deviation values. 
We may not find the general conclusion; 
however, we observed differences across the categories. 
For example, the 25th, 50th, and 75th percentiles of 
the $Img$ and $Mov$ categories in $IssueResolvedTime$ are 
longer than those of the $None$ category. 

% \section{Research Questions}
\label{sec:rqs}

To clarify the characteristics of 
the issues with images and movies, 
we executed an initial analysis 
in our dataset. 
Specifically, we addressed the following 
two research questions. 
\begin{itemize}
	\item[RQ1:] \textbf{\RQone{}}\\
	The feature to share movies on GitHub 
	is still new. 
	Hence, we suppose that using visualization 
	is not popular for developers. 
	In this RQ, we counted the number of 
	issues that use either issues or movies 
	to clarify whether using visualization 
	is popular. 
	\item[RQ2:] \textbf{\RQtwo{}}\\
	We suppose that developers use visualization 
	in particular issues. 
	For example, developers may use visualization 
	to share the way to reproduce bugs. 
	In this RQ, we clarify the differences 
	between the issues with and without 
	visualization. 
\end{itemize}





% We used $Words$ to compute the TFIDF values\masa{need citation} for each category. 
% The TFIDF values will support researchers who investigate 
% the differences in the appearances of words with and without 
% images/movies. 
%% We suppose that images and movies are utilized to 
%% describe specific contents such as GUI bugs. 
%% Hence, we computed the TFIDF values\masa{need citation} 
%% on $Words$ for each category. 
%% The TFIDF values clarify the differences in the appearances of 
%% words with and without images/movies. 



\begin{figure}[t]
\centering
\includegraphics[width=1\linewidth]{./figures/data-category-trend.pdf}
\caption{ 
  The proportions of issues for each category\masa{Is this correct?}
  }
\label{fig:data-cat-trend}
\end{figure}

\subsection{Data Description}
In this section, we show the basic information of 
the collected visual issue reports. \kashiwa{Wright how many issues are used (again)?} 
Specifically, we show the trend of using 
visual issue reports and 
the topics visual issue reports are used to describe.

To clarify the trend, we show the proportion of 
visual issue reports for each year. 

\textbf{Using visual issue reports is still not popular.} 
\fig{fig:data-cat-trend} shows the proportions of 
visual issue reports and issue reports in the $Img$ category 
for each year. 
The y-axis shows the proportion. 
We observed that the proportion of the visual issue reports is 
less than 1\%. 
Even the proportion of the issue reports with images is around 6\% 
in 2020 and 2021. 
Hence, the proportions of the visual issue reports and 
even issue reports in the $Img$ category are still low. 
Given this result, both the images and movies are still not 
popular for developers. 

However, the proportions have slightly increased so far 
except for 2021. 
This may be because we collected the resolved issues 
in 2021. 
Hence, using visual issue reports may be more popular in the future. 

\section{Results}
\label{sec:results}

% 初期段階の調査として,画像及び動画がissueの各attributesに
% 与える影響について調査する.
% 表2で得られたそれぞれのカテゴリ間で比較を比較を行う.
% 等分散性を仮定できないデータが含まれていたため,
% 比較にはSteel-Dwass testを採用する.
% 
% visualizationは,issueの解決時間,
% コメント数,文字数に何らかの影響を与える.
% 検定結果を表6に示す.
% 有意水準は0.05を採用する.
% IssueResolvedTimeにおいて,
% ImgとMovは片側検定で有意差があった.
% また,表Xより,平均値,中央値でNoneカテゴリの
% issueよりImgとMovカテゴリのissueの方が
% 課題解決時間が長かった.
% よって,画像もしくは動画がissue作成時に付与
% される課題はその解決時間が有意に長くなる.
% また,#comments及び#charsでは両側検定で
% 有意であり,これらのattributesにも
% 有意な影響を与えることがわかった.
% ただし,imageとmovieに有意差は
% 観測されなかった.

% As an initial analysis, we investigated 
% the impact of images and movies on 
% the issues in terms of the attributes. 
% As an initial analysis, 


\subsection*{RQ1: \RQone{}}
\begin{figure}[t]
    \centering
    %\includegraphics[width=0.5\linewidth]{tex-kondo/figures/words.png}
    \includegraphics[width=0.5\linewidth]{./figures/words.png}
    \caption{Distributions of words written in issue reports. }
    \label{fig:words}
\end{figure}
\fig{fig:data-cat-trend} shows the distributions in the number of words written in descriptions of issue reports. The green line in the box shows the median, the bottom and the top lines of the box show 25 and 75 percentile, respectively. The median of \#word was 36 in Img, 46 in Vid, and 36 in None. The number of words in videos is slightly larger than the others but no statistically significant differences are observed. This implies that reporters write as many texts as the others to describe the contents of videos. 

\summarybox{Answer to RQ1}
{\bf No, visual issue reports still require reporters to write similar amounts of texts to describe bugs. 
}

\subsection*{RQ2: \RQtwo{}}
\begin{figure}[t]
    \centering
    %\includegraphics[width=1\linewidth]{tex-kondo/figures/discussions.png}
    \includegraphics[width=1\linewidth]{./figures/discussions.pdf}
    \caption{Amount of texts written in issue reports. }
    \label{fig:words}
\end{figure}

\textbf{The values of visual issue reports 
result in faster and larger than the other issue reports 
in terms of $FirstCommentTime$ and $\#comments$.
} 
%However, we do not observe significant differences.
% \tab{tab:issue_stat_categories} shows the statistics
% of the collected issues for the $Img$, $Mov$,
% and $None$ categories.
\fig{fig:words}
shows the values of the Discussion dimension
for each category. 
% The Mean row indicates the average values; 
% the Min and Max rows indicate the minimum and maximum values; 
% the 25th, 50th, and 75th rows indicate the percentile values; 
% the S.D. row indicates the standard deviation values. 
We observed that the 25th, 50th, and 75th percentiles of 
the $Mov$ category in 
this dimension
%$FirstCommentTime$ and $\#comments$ 
result in the largest or faster values than those of 
the other two categories.
Hence, the visual issue reports get more responses 
while they get faster responses. 


%However, we do not observe significant differences 
\textbf{
However, we occasionally observe non-significant differences
in $FirstCommentTime$ and $\#comments$.}
%\textbf{
%However, the visual issue reports are not significantly different 
%from the issue reports in the Img category
%in $FirstCommentTime$ and $\#comments$.
%}
\tab{tab:Steel-Dwass-test} shows the results of 
the Steel-Dwass test. 

The asterisks indicate significance based on 
the Steel-Dwass test: * indicates $p$ < 0.05 in 
the two-sided test; 
** indicates $p$ < 0.05 in the one-sided test. 
%In summary, the visual issue reports do not show 
%significant differences compared with 
%the issue reports with images in 
%In summary, the visual issue reports are not 
%significantly different from the issue reports 
%in the Img category, 
%whereas visual issue reports and 
%the issue reports in the Img category 
%are significantly different from the issue reports 
%in the None category in
%$FirstCommentTime$ and $\#comments$.
In summary, the visual issue reports are not 
significantly different in $FirstCommentTime$ 
compared with the other two categories and 
are not significantly different in $\#comments$ 
compared with the Img category.

\summarybox{Summary of RQ2}
{\bf Yes, visual issue reports are likely to receive more comments than the others but the time until the first response is not different. 
}

\begin{table}[t]
  \begin{center}
  \caption{The Steel-Dwass test results \masa{finally we should remove this table}}
  \begin{tabular}{l r|r}
    \toprule
    Attributes & Category & $p$-value\\
    \midrule
     & \bf{$Img$~vs~$None$} & **~0.002 \\
     $IssueResolvedTime$ & \bf{$Mov$~vs~$None$} & **~0.021 \\
     & \bf{$Img$~vs~$~Mov$} & 0.381 \\
    \midrule
     & \bf{$Img$~vs~$None$} & 0.764 \\
     $FirstCommentTime$ & \bf{$Mov$~vs~$None$} & 0.351 \\
     & \bf{$Img$~vs~$~Mov$} & 0.404 \\
    \midrule
     & \bf{$Img$~vs~$None$} & *~0.001 \\
     $\#comments$ & \bf{$Mov$~vs~$None$} & *~0.001 \\
     & \bf{$Img$~vs~$~Mov$} & 0.211 \\
    \midrule
     & \bf{$Img$~vs~$None$} & 0.488 \\
     $\#words$  & \bf{$Mov$~vs~$None$} & 0.501 \\
     & \bf{$Img$~vs~$~Mov$} & 0.354 \\
    \bottomrule
  \end{tabular}\\
  %\small
  % *~ : significant in two-sided test \\
  % ** : significant in one-sided test \\
  \label{tab:Steel-Dwass-test}
  \end{center}
\end{table}








\subsection*{RQ3: \RQthree{}}
\begin{figure}[t]
    \centering
    %\includegraphics[width=0.6\linewidth]{tex-kondo/figures/fixes.png}
    \includegraphics[width=0.6\linewidth]{./figures/fixes.pdf}
    \caption{Amount of texts written in issue reports. 
    The median of \#word was 2.96 in Img, 3.01 in Vid, and 1.09 in None.
    \masa{I moved this comment to here (maybe removed later? I'm not sure)}
    }
    \label{fig:resolvedtime}
\end{figure}
%% We investigated 
%% the differences with and without images and movies 
%% for each attribute on the issues. 
%% Specifically, we compared the attributes between 
%% the categories in \tab{tab:issue-category}. 
%% Because our preliminary study shows that 
%% the distributions for each category do not 
%% come from normal distributions, 
%% we used a non-parametric test called the \textit{Steel-Dwass test}. 

%\textbf{The visual issue reports get resolved faster than 
\textbf{The visual issue reports need a longer time to be resolved than
the issue reports in the None category. }
% \textbf{The attribute values are changed 
% with and without images and movies.}
% \tab{tab:Steel-Dwass-test} shows the results of 
% the Steel-Dwass test. 
% The asterisks indicate significance based on 
% the Steel-Dwass test: * indicates $p$ < 0.05 in 
% the two-sided test; 
% ** indicates $p$ < 0.05 in the one-sided test. 
We observed that the issues in the $Img$ category and 
the $Vid$ category are significantly different from those 
in the $None$ category in terms of $IssueResolvedTime$
(\tab{tab:Steel-Dwass-test}). 
% \tab{tab:issue_stat_categories} shows the statistics
% of the collected issues for the $Img$, $Mov$,
% and $None$ categories.
% The Mean row indicates the average values; 
% the Min and Max rows indicate the minimum and maximum values; 
% the 25th, 50th, and 75th rows indicate the percentile values; 
% the S.D. row indicates the standard deviation values. 
% % We may not find the general conclusion; 
% % however, we observed differences across the categories. 
% % For example, the 25th, 50th, and 75th percentiles of 
% % the $Img$ and $Mov$ categories in $IssueResolvedTime$ are 
% % longer than those of the $None$ category. 
% % % \tab{tab:issue_stat_categories} shows that 
%This table shows that 
Also, contrary to our hypothesis, \fig{fig:resolvedtime} and our observation show that 
the mean and the median $IssueResolvedTime$ values of 
the issue reports in the $Img$ category and the $Vid$ category are 
longer than those in the $None$ category. 
Hence, the issue reports with images or videos tend to need 
a longer time to be resolved. 

% In addition, we observed that the issues in 
% the $Img$ category and the $Mov$ category are 
% significantly different from those in 
% the $None$ category in terms of $\#comments$ and 
% $\#chars$. 
% Hence, the issues with images or movies 
% tend to have different numbers of 
% comments and words in the issue description.

% It should be noted that we do not observe 
% the difference between the $Img$ category and 
% the $Mov$ category. 


%{\bf No, resolution time for issues with videos are not different from those for the others. 
\summarybox{Summary of RQ3}
{\bf No, resolution time for the visual issue reports needs a longer time than the issue reports in the None category at least.
}
\begin{table*}[t]
  \begin{center}
  \caption{The statistics of the collected issues for each category}
  % \scalebox{0.85}[0.85]{
  \begin{tabular}{l c c c| c c c| c c c| c c c} 
    \toprule
    & \multicolumn{3}{c}{$IssueResolvedTime$} & \multicolumn{3}{c}{$FirstCommentTime$} & \multicolumn{3}{c}{$\#comments$} & \multicolumn{3}{c}{$\#chars$}\\
    & \textbf{$Img$} & \textbf{$Mov$} & \textbf{$None$} & \textbf{$Img$} & \textbf{$Mov$} & \textbf{$None$} & \textbf{$Img$} & \textbf{$Mov$} & \textbf{$None$} & \textbf{$Img$} & \textbf{$Mov$} & \textbf{$None$} \\ 
    \midrule
    Mean & 24.86 & 21.77 & 21.14 & 6.83 & 7.31 & 9.15 &  4.04 & 5.56 & 2.29 &  995.2 & 844.8 & 1350.6 \\
    Min & 30sec & 38sec & 30sec & 1sec & 1sec & 1sec &  0 & 0 & 0 &  6 & 1 & 0 \\
    1rd Q & 10.2hrs & 15.7hrs & 2.14hrs &  10.6min & 3.38min & 10.4min &  1 & 1 & 0 &  209 & 237 & 59 \\
    Mdn & 2.96 & 3.01 & 1.09 &  3.26hrs & 2.4hrs & 3.0hrs &  2 & 3 & 1 &  449 & 468 & 251 \\
    3rd Q & 17.30 & 13.12 & 10.80  & 1.38 & 1.18 & 1.84 &  5 & 7 & 3 &  901 & 897 & 880 \\
    Max & 364.56 & 361.83 & 364.99 &  364.10 & 337.61 & 364.61 &  283 & 120 & 439 &  $2.60E5$ & $5.89E4$ & $2.61E5$ \\
    S.D. & 55.64 & 53.96 & 53.64 &  28.27 & 32.77 & 34.39 &  6.873 & 8.710 & 4.619 &  $4.14E3$ & $2.23E3$ & $4.81E3$ \\
    \bottomrule
  \end{tabular}
  % }
  \label{issue_class_data}
  \end{center}
\end{table*}


%\section{Discussion}
\begin{table}[t]
  \begin{center}
  % \caption{The top-10 words in terms of TFIDF}
  \caption{Top-10 words in terms of TFIDF}
  \begin{tabular}{r | c c c}
    \hline
     & $Img$ & $Vid$ & $None$\\
    \hline
    1 & image & when & dependabot\\
    2 & screenshot & dropdown & code\\
    3 & when & view & file\\
    4 & error & package & pullrequest\\
    5 & screen & issue & version\\
    6 & version & python & error\\
    7 & shot & height & use\\
    8 & file & react & lib\\
    9 & test & button & add\\
    10& code & text & commit\\
    \hline
  \end{tabular}\\
  \label{tab:tfidf-result}
  \end{center}
\end{table}

\subsection{Why Do Issue Reports with Videos or Images Take a Longer Time than Others?}
To clarify the topics, we computed the TFIDF values 
on $Words$ for each category. 
The TFIDF values clarify the differences in the appearances of 
words between visual issue reports and 
non-visual issue reports. 
\textbf{The visual issue reports and the issues 
reports with images have more words 
about visualization or GUI.}
\tab{tab:tfidf-result} shows the top-10 TFIDF words
% in $Words$ 
for each category.
It should be noted that we remove a kind of stop words such as 
``at'', ``it'', and ``the''. 
% extracted from our dataset. 
We observed words related to visualization such as 
``image'' and ``view'' in the $Img$ and $Mov$ categories. 
Also, we observed words related to GUI such as 
``dropdown'' and ``button'' in the $Mov$ category. 
The word ``when'' indicates high TFIDF values 
in the $Img$ and $Mov$ categories. 
Our manual analysis reveals that the issues with 
these words relate to reporting bugs. 
Hence, the issues in these two categories are 
probably more related to reporting bugs. 
\section{Discussion}
\begin{table}[t]
  \begin{center}
  % \caption{The top-10 words in terms of TFIDF}
  \caption{Top-10 words in terms of TFIDF}
  \begin{tabular}{r | c c c}
    \hline
     & $Img$ & $Vid$ & $None$\\
    \hline
    1 & image & when & dependabot\\
    2 & screenshot & dropdown & code\\
    3 & when & view & file\\
    4 & error & package & pullrequest\\
    5 & screen & issue & version\\
    6 & version & python & error\\
    7 & shot & height & use\\
    8 & file & react & lib\\
    9 & test & button & add\\
    10& code & text & commit\\
    \hline
  \end{tabular}\\
  \label{tab:tfidf-result}
  \end{center}
\end{table}

\subsection{Why Do Issue Reports with Videos or Images Take a Longer Time than Others?}
To clarify the topics, we computed the TFIDF values 
on $Words$ for each category. 
The TFIDF values clarify the differences in the appearances of 
words between visual issue reports and 
non-visual issue reports. 
\textbf{The visual issue reports and the issues 
reports with images have more words 
about visualization or GUI.}
\tab{tab:tfidf-result} shows the top-10 TFIDF words
% in $Words$ 
for each category.
It should be noted that we remove a kind of stop words such as 
``at'', ``it'', and ``the''. 
% extracted from our dataset. 
We observed words related to visualization such as 
``image'' and ``view'' in the $Img$ and $Mov$ categories. 
Also, we observed words related to GUI such as 
``dropdown'' and ``button'' in the $Mov$ category. 
The word ``when'' indicates high TFIDF values 
in the $Img$ and $Mov$ categories. 
Our manual analysis reveals that the issues with 
these words relate to reporting bugs. 
Hence, the issues in these two categories are 
probably more related to reporting bugs. 
\section{Limitation}
\label{sec:limitation}



\section{Related Work}
\label{sec:relate}

Ford~\et~\citep{ford2021TOSEM} conducted 
two large-scale surveys to study the affection of 
the situation developers work from home. 
They figure out the advantages and disadvantages of 
remote work. 
Interestingly, the change by remote work causes opposite opinions 
to developers. 


% \section{Future Work}
\label{sec:future}

We currently plan to further improve our dataset. 
In this section, we describe ideas to improve the dataset. 
First, we intend to remove the bot comments as much as 
possible. 
Almost all OSSs use bots to promote software development 
such as automatically closing the abandoned issues with 
a comment. 
However, such bots may change the characteristics of 
the dataset. 
Hence, we should remove such comments. 
Also, we want to classify the data based on 
the characteristics of the repositories. 
Currently, we collected all repositories. 
However, the repositories have their characteristics. 
A repository is a web framework while another repository is 
a machine learning library. 
In addition, the development state of repositories is different. 
The characteristics of the repositories in which 
developers start to develop and the repositories that 
have already released many major versions should be different. 
Finally, we intend to improve the script and 
collect all repositories that meet the condition we described in \sec{sec:dataset}. 

\section{Conclusion}
\label{sec:conclusion}

\masa{conclusion here}

Finally, we present future research directions with our dataset. 


\noindent
\textbf{The impact analysis of movies and images on software development.}
GitHub added the feature~\citep{github-video-blog} to easily 
share movies with GitHub to earn advantages 
in software development such as reproducing bugs easily in issues and 
explaining the background of changes in pull requests. 
However, no studies exist that investigate whether such movies 
impact software development. 
Hence, investigating the impact of movies on software development 
is a future research direction. 
Specifically, we study the reduction of issue resolution time, 
the response rate of developers, and 
the reduction of the number of comments 
in which developers write ``works for me''. 

\noindent
\textbf{Automated bug reproduction with image processing.}
Reproducing bugs is a time-consuming process 
in software development\masa{citation}.
Automating this process would support developers 
to quickly find and fix the cause of bugs. 
Hence, it is an important future research direction 
to automate bug reproduction. 
As the evolution of image processing with deep learning models, 
the accuracy of image processing significantly improves\masa{citation}. 
Hence, we apply such deep learning models to movies 
to implement automated bug reproduction tools. 


\section*{Acknowledgment}
% This work has been supported by JSPS KAKENHI Japan
% (Grant Numbers: \masa{update}.
This research was partially supported by \texttt{double\_blind}.


\bibliographystyle{ACM-Reference-Format}
\bibliography{reference}



\end{document} 
