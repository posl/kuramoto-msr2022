\begin{abstract}
Issue reports are a pivotal interface between developers and users to receive the information about bugs present in their products. In practice, issue reports often have incorrect or insufficient information to reproduce bugs, which delays the whole bug-fixing process. 
To facilitate their bug-reproduction work, GitHub has provided a new feature to share movies (e.g., mp4 files). Using movies, reporters can notify of the details of bugs by visualizing the symptoms, reproduction steps, and other kinds of important information. 

While such visualized issue reports (\textit{visual issue reports}) have the potential to significantly improve the bug-fixing process, no studies have empirically examined the impact on the bug-fixing process. In this paper, we conduct a preliminary study to reveal the characteristics of visual issue reports by comparing them with non-visual issue reports.  

We collect 3,819 movies and 33,079 images from 711,160 issues on 4,173 publicly available repositories. Contrary to our hypothesis, our preliminary analysis shows that the visual issue reports need a longer time to be resolved. In addition, we observed that their discussions with visual issue reports contain many topics about conditions for reproduction (e.g., when) and GUI (e.g., dropdown). 

%In addition, to promote the research about visual issue reports,  we provide a dataset of visual issue reports.

% Issue reports are pivotal for developers
% to receive the information of bugs from reporters. 
% % However, it is difficult for inexperienced reporters to report 
% % what developers need with issue reports. 
% % In such cases, issue reports have incomplete or incorrect information 
% % such as the insufficient steps to reproduce bugs. 
% However, issue reports occasionally have incorrect or 
% incomplete information such as insufficient steps to reproduce bugs. 
% % Consequently, developers delay fixing bugs. 
% Such issue reports cause delays in fixing bugs. 
% % To relieve this difficulty, GitHub provides a new feature to 
% To improve such incorrect or incomplete information, 
% GitHub has started to provide a new feature to share movies on GitHub.
% Reporters can visualize issue reports by attaching movies
% such as the steps to reproduce bugs.
% %  as the screen capture. 
% GitHub argued that such visualized issue reports 
% (\textit{visual issue reports}) 
% have the potential to easily share information.
% % such as the steps to reproduce bugs. 
% % Such visualized issue reports (\textit{visual issue reports}) have the potential to 
% % easily share information without incorrect one.  
% % For example, reporters can share the steps to reproduce bugs 
% % with the screen capture and developers can understand 
% % the steps quickly~\citep{github-video-blog}.
% %% In distributed development, asynchronous communication 
% %% across developers is an essential part. 
% %% To promote such communication, 
% %% GitHub has recently released a new feature to share movies on GitHub. 
% %% This feature can visualize various contents users want to share on GitHub. 
% %% Such visualization can support questioner and questionee because 
% %% it is easy to share and understand the contents such as reporting bugs and 
% %% clarifying the way to reproduce the bugs.
% %% % Using visualization is a typical solution to 
% %% % % improve the understandability of the contents 
% %% % support questioner and questionee
% %% % in software development such as showing demos. 
% % Hence, this feature has the potential to make asynchronous communication 
% % across developers quickly. 
% % However, no database or studies about the impact of
% However, no studies about the impact of
% visual issue reports in software development exist so far. 
% In this paper, we conducted a preliminary study to clarify
% the characteristics of visual issue reports by comparing them
% to the issue reports with images and the other issue reports. 
% In addition, to promote the research about visual issue reports, 
% we present a dataset of visual issue reports.
% % repositories on GitHub and show the result of the initial analysis. 
% % In this paper, we present the first database that 
% % consists of movies and images in publicly available 
% % repositories on GitHub and show the result of the initial analysis. 
% Specifically, we collected 3,819 movies and 33,079 images 
% from 711,160 issues on 4,173 publicly available repositories.
% Our preliminary analysis revealed that the visual issue reports
% need a longer time to be resolved and 
% are related to the topics of visualization and GUI. 
\end{abstract}