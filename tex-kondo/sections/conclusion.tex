\section{Conclusion and Future Work}
\label{sec:conclusion}

In this paper, we present a dataset, 
which consists of issues with images and movies 
in GitHub and show the results of our initial analysis. 
This dataset represents the number of images and movies 
for each issue, 
the issue resolution time, 
the number of comments, 
the time until the first comment is submitted, 
the number of characters in the description, 
and the TFIDF of words in the descriptions. 
% We show that the statistics of the dataset and 
% implied that the images and movies may affect 
% the issue resolution time. 
This dataset is a first step to investigate 
the impact of images and movies on communication in GitHub. 
In addition, our initial analysis shows that images and movies 
are used in issues that need a longer time to be resolved and 
are related to the topics of visualization and GUI. 
Hence, future studies are necessary to deeply look at 
the impact of images and movies to accelerate communication 
in GitHub.

Finally, we present future research directions. 

\noindent
\textbf{Improve the quality of the dataset.}
% We currently plan to further improve our dataset. 
% In this section, we describe ideas to improve the dataset. 
Our dataset has space to further improve.
% First, we intend to remove the bot comments as much as 
% possible. 
First, removing the bot comments as much as 
possible is necessary. 
Almost all OSSs use bots to promote software development 
such as automatically closing the abandoned issues with 
a comment. 
However, such bots may change the characteristics of 
the dataset. 
% Hence, we should remove such comments. 
% Also, we want to classify the data based on 
% the characteristics of the repositories. 
Also, classifying the data based on 
the characteristics of the repositories is important. 
Currently, we collected all repositories as one dataset. 
However, the repositories have their characteristics. 
A repository is a web framework while another repository is 
a machine learning library. 
In addition, the development state of repositories is different. 
The characteristics of the repositories in which 
developers start to develop and the repositories that 
have already released many major versions should be different. 
% Finally, we intend to improve the script and 
% collect all repositories that meet the condition we described in \sec{sec:dataset}. 
Finally, collecting all repositories that meet 
the condition described in \sec{sec:dataset} is necessary. 


\noindent
\textbf{The impact analysis of movies and images on software development.}
GitHub added the feature~\citep{github-video-blog} to easily 
share movies with GitHub to earn advantages 
in software development such as reproducing bugs easily in issues and 
explaining the background of changes in pull requests. 
However, no studies exist that investigate whether such movies 
impact software development. 
Hence, investigating the impact of movies on software development 
is a future research direction. 
% Specifically, we study the reduction of issue resolution time, 
% the response rate of developers, and 
% the reduction of the number of comments 
% in which developers write ``works for me''. 

\noindent
\textbf{Automated bug reproduction with image processing.}
Reproducing bugs is a time-consuming process 
in software development\masa{citation}.
Automating this process would support developers 
to quickly find and fix the cause of bugs. 
Hence, it is an important future research direction 
to automate bug reproduction. 
As the evolution of image processing with deep learning models, 
the accuracy of image processing significantly improves\masa{citation}. 
Hence, we apply such deep learning models to movies 
to implement automated bug reproduction tools. 