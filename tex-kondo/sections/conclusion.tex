\section{Conclusion}
\label{sec:conclusion}

In this paper, we present a dataset, 
which consists of issues with images and movies 
in GitHub. 
This dataset represents the number of images and movies 
for each issue, 
the issue resolution time, 
the number of comments, 
the time until the first comment is submitted, 
the number of characters in the description, 
and the TFIDF of words in the descriptions. 
We showed the statistics of the dataset and 
implied that the images and movies may affect 
the issue resolution time. 
This dataset is a first step to investigate 
the impact of images and movies on communication in GitHub. 

Finally, we present future research directions with our dataset. 


\noindent
\textbf{The impact analysis of movies and images on software development.}
GitHub added the feature~\citep{github-video-blog} to easily 
share movies with GitHub to earn advantages 
in software development such as reproducing bugs easily in issues and 
explaining the background of changes in pull requests. 
However, no studies exist that investigate whether such movies 
impact software development. 
Hence, investigating the impact of movies on software development 
is a future research direction. 
Specifically, we study the reduction of issue resolution time, 
the response rate of developers, and 
the reduction of the number of comments 
in which developers write ``works for me''. 

\noindent
\textbf{Automated bug reproduction with image processing.}
Reproducing bugs is a time-consuming process 
in software development\masa{citation}.
Automating this process would support developers 
to quickly find and fix the cause of bugs. 
Hence, it is an important future research direction 
to automate bug reproduction. 
As the evolution of image processing with deep learning models, 
the accuracy of image processing significantly improves\masa{citation}. 
Hence, we apply such deep learning models to movies 
to implement automated bug reproduction tools. 