\section{Dataset}
\label{sec:dataset}

In this section, we describe the data collection process and 
the overview of the collected dataset. 

\subsection{Data Collection}
We collected the data from the issues of 
the publicly available repositories on GitHub. 
We first select the repositories based on the following conditions:
\begin{itemize}
	\item the number of stars $\geq$ 10
	\item the number of issues $\geq$ 1
	\item the latest commit\masa{or the latest issue?} is in 2021
\end{itemize}
We used the number of stars for retrieving the repositories 
in which the owner may communicate with other developers and 
discard repositories developed by only the owner. 
In addition, we used the date of the latest commit 
because the major release of the feature is in 2021. 
Beforehand, the feature was the beta version. 
The number of the repositories that meet 
the conditions is 289,115. 
We randomly selected 4,173 repositories from them 
due to the execution time to collect issues. 
This number is less than 2\% of all repositories. 
However, the number of resolved issues 
in these repositories is already 770,656. 
We discuss the threat to validity of 
this process in \sec{sec:future}. 
This process was conducted from November 2021 to December 2021.


\begin{table}[t]
    \begin{center}
    \caption{The attributes we collected from the issues}
    \scalebox{0.85}[0.85]{
    \begin{tabular}{ll} 
        \toprule
        \multicolumn{1}{c}{\textbf{Attributes}} & \multicolumn{1}{c}{\textbf{Description}} \\ 
        \midrule
        $IssueResolvedTime$ & The time until the issue is resolved (day) \\
        $FirstCommentTime$ & The time until the first comment (day) \\
        $\#comments$ & The number of comments \\
        $\#chars$ & \masa{im not sure what is this} \\
        $\#imgs$ & \# of attached images when the issue is created \\
        $\#movs$ & \# of attached movies when the issue is created \\
        $\#words$ &  \masa{im not sure what is this} \\
        $IssueCreatedYear$ & The year when the issue is created \\
        \bottomrule
    \end{tabular}
    }
    \label{tab:issue-attr}
    \end{center}
\end{table}


We retrieved the attributes from the collected issues 
that are the data in the database. 
\tab{tab:issue-attr} shows the eight attributes 
retrieved from the issues. 

We used \texttt{PyGithub}\masa{add citation} to execute GitHub API 
to retrieve the attributes from the issues. 
$IssueOpenTime$, $FirstCommentTime$,
and $IssueCreatedYear$ can be directly
retrieved by \texttt{PyGithub}.
On the contrary,
$\#chars$, $\#imgs$, $\#movs$, and $\#words$
need the conversion from the retrieved raw data.
We describe the details of the conversion. 
The attached images and movies are transformed into 
URLs and put in the text of issues as the markdown format. 
The following URL is an example.

\begin{quote}
	https://user-images.githubusercontent.com/XXX.mp4
\end{quote}

\noindent{}
The part of XXX consists of alphanumeric, ``/'', and ``-''.
Hence, we prepared the regular expression for this URL and 
count the appearances of images and movies as $\#imgs$ and $\#movs$. 
The identification of images and movies is based on the extension of 
the URLs. 

For the issues that include the URL, we excluded them and 
counted the number of words as $\#words$. 
In addition, we counted the number of characters as $\#chars$. 

We noticed that some $IssueOpenTime$ values are less than zero. 
We decided that these values are invalid and the issues having 
these values are excluded from the dataset. 
Specifically, we only retrieved the issues that meet 
the condition: $30\ sec \leq IssueOpenTime \leq 1\ year$.
The number of issues that meet this condition is 711,160 (92.23\%).

\subsection{Dataset Description}
\masa{Originality and easy analysis (if need)}

