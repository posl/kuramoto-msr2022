\section{Discussion}
%\begin{table}[t]
  \begin{center}
  % \caption{The top-10 words in terms of TFIDF}
  \caption{Top-10 words in terms of TFIDF}
  \begin{tabular}{r | c c c}
    \hline
     & $Img$ & $Vid$ & $None$\\
    \hline
    1 & image & packages & file\\
    2 & error & view & error\\
    3 & screenshot & when & lib\\
    4 & when & python & if\\
    5 & have & pageviewcontroller & line\\
    6 & if & config & java \\
    7 & version & local & get\\
    8 & get & version & have\\
    9 & using & problem & when\\
    10& file & error & version\\
    \hline
  \end{tabular}\\
  \label{tab:tfidf-result}
  \end{center}
\end{table}

% \subsection{Why Do Issue Reports with Videos or Images Take a Longer Time than Others?}
\subsection{Why Are Resolution Times of Visual Issue Reports Unchanged from Others?}
In this study, we observed that developers write fewer words in visual issue reports compared to non-visual issue reports (RQ1). 
However, we found that visual issue reports do not lead to active discussions (RQ2).
Also, resolution times are unchanged (RQ3), which rejected our hypothesis. 
To investigate the reason why visual issue reports take the same time to be closed, we examine the differences in the contents of bugs in this section.

We extracted words from the descriptions of closed issue reports in the dataset, and  removed stop words such as ``at'', ``it'', and ``the'' from them. 
Then, we calculated TF-IDF values~\citep{salton1988-tfidf} to clarify the characteristic words for each types of issues (i.g., Vid, Img, None). 

\tab{tab:tfidf-result} shows the top-10 characteristic words in each category, calculated by TF-IDF.
First, in the $Vid$ and $Img$ categories, we observed many words related to GUI such as ``screenshot'' and ``pageviewcontroller''. 
On the other hand, ``line'' and ``java'' related to source codes are located at the top of $None$.
Second, it is worth noting that ``when'' and ``error'' are located at the top rank of the lists in all categories.

These findings imply that reporters explain different contents in the visual issue reports and the non-visual issue reports, whereas both explain steps to reproduce errors. 
Hence, we suppose that texts in the description are different because of different contents but resolution times to fix errors are unchanged. 



\subsection{Are Visual Issue Reports and Non-Visual Issues Used for Similar Aims?}
In this study, we observed that developers write fewer words in issue reports with images ($Img$) compared to non-visual issue reports ($None$). On the other hands, between $None$ and $Vid$, no statistically significant difference is observed (RQ1). 
Also, we confirmed that there are no statistical significant differences of the resolution time between visual issue reports (i.e., $Img$ and $Vid$) and $None$ (RQ3). These findings rejected our hypothesis. 
To investigate these reasons, we examine the differences in the contents of bugs in this section.

We extracted words from the descriptions of closed issue reports in the dataset, and  removed stop words such as ``at'', ``it'', and ``the'' from them. 
Then, we calculated TF-IDF values~\citep{salton1988-tfidf} to clarify the characteristic words for each types of issues (i.g., Vid, Img, None). 

\tab{tab:tfidf-result} shows the top-10 characteristic words in each category, calculated by TF-IDF.
First, in the $Vid$ and $Img$ categories, we observed many words related to GUI such as ``screenshot'' and ``pageviewcontroller''. 
On the other hand, ``line'' and ``java'' related to source codes are located at the top of $None$.
Second, it is worth noting that ``when'' is shown in all categories but it is located at the top rank of the lists in $Vid$ and $Img$. This implies that visual issue reports are utilized to describe conditions/steps to reproduce bugs. In particular, as ``config'' is shown only in $Vid$, videos may be used to explain complicated conditions/environment to reproduce bugs. This studies do not measure the degree of the difficulties in reproducing bugs but future work should investigate it.

\begin{table}[t]
  \begin{center}
  % \caption{The top-10 words in terms of TFIDF}
  \caption{Top-10 words in terms of TFIDF}
  \begin{tabular}{r | c c c}
    \hline
     & $Img$ & $Vid$ & $None$\\
    \hline
    1 & image & packages & file\\
    2 & error & view & error\\
    3 & screenshot & when & lib\\
    4 & when & python & if\\
    5 & have & pageviewcontroller & line\\
    6 & if & config & java \\
    7 & version & local & get\\
    8 & get & version & have\\
    9 & using & problem & when\\
    10& file & error & version\\
    \hline
  \end{tabular}\\
  \label{tab:tfidf-result}
  \end{center}
\end{table}


\subsection{Future Research Direction}
This section discusses what should be considered by future studies. 

\noindent
\textbf{Fine-grained analysis. }
% In RQ2, we showed that visual issue reports are 3-times likely to receive more comments than non-visual issues. As these might be caused by that  issue reports attract many developers or that visual issue reports deal with difficult problems. 
In RQ2, we showed that issue reports in Vid take longer times to receive the first comment than issue reports in Img. 
As these might be caused by that issue reports with images attract more developers or that issue reports with videos are more difficult problems.
Future studies should examine how many developers are involved~\citep{DBLP:conf/icsm/BavotaR15}, severity tags~\citep{DBLP:conf/issre/ZhouNG15}, and size of changes~\citep{DBLP:conf/kbse/HattoriL08}.

In this study, we studied only closed bugs and examined only resolution time in RQ3 (Fix). However, previous studies examined several statuses of bugs. For example, Joorabchi et al.~\citep{DBLP:conf/msr/JoorabchiMM14} studied ``Works For Me'', Shihab et al.~\citep{DBLP:journals/ese/ShihabIKIOAHM13} studied reopen bugs, and Zou~\et~\citep{DBLP:conf/compsac/ZouXZCL15} examined bug fixing rate (\eg, ``Won't fix''). However, most of the studies use other bug tracking systems, Bugzilla~\citep{Bugzilla} or Jira~\citep{JIRA}. These bug tracking systems have various resolution statuses such as ``Won't Fix'' and ``Works For Me''  in default but GitHub we studied does not. Developers on GitHub can provide tags but it is not common (\masa{X}\%). Future studies should collect more issues and show the percentage of each status, etc. 

\noindent
\textbf{Bug reproduction Automation.}
Developers often suffer from reproducing bugs with the reported information~\citep{DBLP:conf/sigsoft/ChaparroLZMPMBN17}\citep{DBLP:conf/icsm/0001KC20}\citep{zimmermann2010TSE}.
Automating this process would support developers to quickly find and fix the cause of bugs. 
Our final goal of this study is to automate bug reproduction. 
We believe that we can make use of image processing technique ~\citep{DBLP:conf/icse/Bernal-Cardenas20}\citep{he2019arxiv}\citep{DBLP:conf/nips/KrizhevskySH12}, using the uploaded videos, in order to identify which pages/screens of systems was being used and what actions was done by users (e.g., which button was clicked). 
This approach would reduce efforts for evaluating if reported issues can be reproducable. 
% \kashiwa{To Masa: Please check if the citation is appropriate for the modified sentenses. }


%Kashiwa: I commented out the following things because this is a bit small topic. I want to talk about future direction here. 

% \textbf{Improve the quality of the dataset.}
% Our dataset has space to further improve.
% First, removing the bot comments as much as 
% possible is necessary. 
% Almost all OSSs use bots to promote software development 
% such as automatically closing the abandoned issues with 
% a comment. 
% However, such bots may change the characteristics of 
% the dataset. 
% Also, classifying the data based on 
% the characteristics of the repositories is important. 
% Currently, we collected all repositories as one dataset. 
% However, the repositories have their characteristics. 
% A repository is a web framework while another repository is 
% a machine learning library. 
% In addition, the development state of repositories is different. 
% The characteristics of the repositories in which 
% developers start to develop and the repositories that 
% have already released many major versions should be different. 
% Finally, collecting all repositories that meet 
% the condition described in \sec{sec:dataset} is necessary. 


% \noindent
% \textbf{The impact analysis of movies and images on software development.}
% GitHub added the feature~\citep{github-video-blog} to easily 
% share movies with GitHub to earn advantages 
% in software development such as reproducing bugs easily in issues and 
% explaining the background of changes in pull requests. 
% However, no studies exist that investigate whether such movies 
% impact software development. 
% Hence, investigating the impact of movies on software development 
% is a future research direction. 
%  

