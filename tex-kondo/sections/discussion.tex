\section{Discussion}
%\begin{table}[t]
  \begin{center}
  % \caption{The top-10 words in terms of TFIDF}
  \caption{Top-10 words in terms of TFIDF}
  \begin{tabular}{r | c c c}
    \hline
     & $Img$ & $Vid$ & $None$\\
    \hline
    1 & image & packages & file\\
    2 & error & view & error\\
    3 & screenshot & when & lib\\
    4 & when & python & if\\
    5 & have & pageviewcontroller & line\\
    6 & if & config & java \\
    7 & version & local & get\\
    8 & get & version & have\\
    9 & using & problem & when\\
    10& file & error & version\\
    \hline
  \end{tabular}\\
  \label{tab:tfidf-result}
  \end{center}
\end{table}

\begin{table}[t]
  \begin{center}
  % \caption{The top-10 words in terms of TFIDF}
  \caption{Top-10 words in terms of TFIDF}
  \begin{tabular}{r | c c c}
    \hline
     & $Img$ & $Vid$ & $None$\\
    \hline
    1 & image & packages & file\\
    2 & error & view & error\\
    3 & screenshot & when & lib\\
    4 & when & python & if\\
    5 & have & pageviewcontroller & line\\
    6 & if & config & java \\
    7 & version & local & get\\
    8 & get & version & have\\
    9 & using & problem & when\\
    10& file & error & version\\
    \hline
  \end{tabular}\\
  \label{tab:tfidf-result}
  \end{center}
\end{table}

\subsection{Why Do Issue Reports with Videos or Images Take a Longer Time than Others?}

\kashiwa{Dependabot makes resolution time small. Future reseach have to check the size of changes.}

We hypothesize that the topics of the visual issue reports 
affect the resolution time. 
To clarify the topics, we computed the TFIDF values 
%on $Words$ for each category. 
on the words in the description for each category. \kashiwa{do not use Words. Just from description}

The TFIDF values clarify the differences in the appearances of 
words between visual issue reports and 
non-visual issue reports. 

%\textbf{The visual issue reports and the issues 
%reports with images have more words 
%about visualization or GUI.}
\tab{tab:tfidf-result} shows the top-10 TFIDF words
% in $Words$ 
for each category.
It should be noted that we remove a kind of stop words such as 
``at'', ``it'', and ``the''. 
% extracted from our dataset. 
We observed words related to visualization such as 
``image'' and ``view'' in the $Img$ and $Vid$ categories. 
Also, we observed words related to GUI such as 
``dropdown'' and ``button'' in the $Vid$ category. 
The word ``when'' indicates high TFIDF values 
in the $Img$ and $Vid$ categories. 
Our manual analysis reveals that the issues with 
these words relate to reporting bugs. 
Hence, the issues in these two categories are 
probably more related to reporting bugs. 
On the contrary, the bot comment ``dependabot'' is 
the top-1 word in the $None$ category. 
\summarybox
{\bf  
%Reporting and fixing the bugs of visualization and GUI 
%may need a longer time. 
One reason is that the bugs of visualization and GUI are 
reported in the issue reports with videos or images 
while the bot comment is in the $None$ category.  
}


\subsection{Future Research Direction}
\kashiwa{TODO: Write future work here}\\

\noindent
\textbf{Cause and Effect. }
As mentined above, issues with videos prolong the time to be resolved. It might be that reporters use videos because it is hard to describe. But it is not sure. 

\noindent
\textbf{Fine-grained analysis. }
%Specifically, we study the reduction of issue resolution time, 
% the response rate of developers, and 
% the reduction of the number of comments 
% in which developers write ``works for me''.
This preliminary study examined only time, comments, etc. Videos will impact seveal factors. For example, works for me. AAA et al. studied works for me. Shihab studied Reopen bugs. However, these tags are exist in Bugzilla  but GitHub does not. We need a manual inspections for finding such tags. 
\\
AAA studied Resolution rate. But likewise, GitHub does not. We could use wont'fix. 


\noindent
\textbf{Bug reproduction Automation.}
Reproducing bugs is a time-consuming process 
in software development\masa{citation}.
Automating this process would support developers 
to quickly find and fix the cause of bugs. 
Hence, it is an important future research direction 
to automate bug reproduction. 
As the evolution of image processing with deep learning models, 
the accuracy of image processing significantly improves\masa{citation}. 
Hence, we apply such deep learning models to movies 
to implement automated bug reproduction tools. 


%Kashiwa: I commented out the following things because this is a bit small topic. I want to talk about future direction here. 

% \textbf{Improve the quality of the dataset.}
% Our dataset has space to further improve.
% First, removing the bot comments as much as 
% possible is necessary. 
% Almost all OSSs use bots to promote software development 
% such as automatically closing the abandoned issues with 
% a comment. 
% However, such bots may change the characteristics of 
% the dataset. 
% Also, classifying the data based on 
% the characteristics of the repositories is important. 
% Currently, we collected all repositories as one dataset. 
% However, the repositories have their characteristics. 
% A repository is a web framework while another repository is 
% a machine learning library. 
% In addition, the development state of repositories is different. 
% The characteristics of the repositories in which 
% developers start to develop and the repositories that 
% have already released many major versions should be different. 
% Finally, collecting all repositories that meet 
% the condition described in \sec{sec:dataset} is necessary. 


% \noindent
% \textbf{The impact analysis of movies and images on software development.}
% GitHub added the feature~\citep{github-video-blog} to easily 
% share movies with GitHub to earn advantages 
% in software development such as reproducing bugs easily in issues and 
% explaining the background of changes in pull requests. 
% However, no studies exist that investigate whether such movies 
% impact software development. 
% Hence, investigating the impact of movies on software development 
% is a future research direction. 
%  

