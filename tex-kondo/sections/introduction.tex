\section{Introduction}
\label{sec:intro}
~{\it ``What makes good issue reports?''} has been studied for decades and is still an ultimate research question for many studies improving the quality of issue reports\citep{TODO}. Issue reports (a.k.a. bug reports) often lack information to reproduce bugs~\citep{DBLP:conf/msr/JoorabchiMM14}\citep{DearGitHub}. 

For example, Zimmermann~\et~\citep{zimmermann2010TSE} reported  that  stack traces and steps to reproduce are considered to be helpful by developers. On the other hand, it is difficult for users to provide these information, which are often missing or incorrect. 
This mismatch between what developers need and what reporters can provide would delay fixing bugs~\citep{DBLP:conf/msr/JoorabchiMM14}. In addition, many studies reported the quality of issue reports impact the issue resolution time~\citep{DBLP:conf/cscw/BreuPSZ10}\citep{DBLP:conf/icse/GuoZNM10} and issue resolution rate~\citep{DBLP:conf/compsac/ZouXZCL15}\citep{DBLP:conf/icse/ZimmermannNGM12}. \kashiwa{These citations need to be confirmed whether they mentioned clearly}

\kashiwa{WIP}
To address this challenge, GitHub released a new feature to easily share movies in May 2021~\citep{github-video-blog}. Developers and reporters can share their movies such as the mp4 format on GitHub (\eg,  issue reports, pull requests, and discussions). 
Visualized issue reports by this feature (\textit{visual issue reports}) have the potential to easily share information without incorrect ones.  
For example, reporters can share the steps to reproduce bugs with the screen capture and developers can understand the steps quickly~\citep{github-video-blog}. 
However, no studies about the impact of visual issue reports exist. Our ultimate goal is to establish the advantages of visual issue reports on software development and promote using them.


As a precursor to this goal, in this paper, 
we conducted a preliminary study about the characteristics of 
visual issue reports in software development on GitHub
by comparing them to the issue reports with images and 
the other issue reports. 
In addition, to promote the research about visual issue reports, 
we built a dataset of visual issue reports. 
This dataset consists of movies and images 
in publicly available repositories on GitHub. 
Specifically, we collected 
3,819 movies and 
33,079 images from 
711,160 issue reports on
4,173 publicly available repositories.

Our initial analysis reveals that 
1) the visual issue reports are still not popular,
while the number of them has slightly increased in recent years, 
2) the visual issue reports need 
a longer time to be resolved than the issue reports without images and movies, and 
3) the visual issue reports are more related to 
the topics of visualization and GUI. 
The main contribution of this paper is 
opening a new research perspective to study visual issue reports
and preparing a new dataset for this research. 