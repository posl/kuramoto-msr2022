\section{Introduction}
\label{sec:intro}
The question "What makes a good issue report?'' has been studied for decades and is still the ultimate research question for many studies aiming to improve the quality of issue reports~\citep{DBLP:conf/icse/HerzigJZ13}\citep{zimmermann2010TSE}\citep{DBLP:conf/eclipse/BettenburgJSWPZ07}. Issue reports (a.k.a. bug reports) often lack the information necessary for developers to reproduce bugs~\citep{DBLP:conf/msr/JoorabchiMM14}\citep{DearGitHub}. 
For example, Zimmermann~\et~\citep{zimmermann2010TSE} report that stack traces and steps for reproducing a bug are considered to be helpful by developers. But, it is difficult for users to provide this information, and it is often missing or incorrect. 
This mismatch between what developers need and what reporters can provide can often delay the fixing of bugs~\citep{DBLP:conf/msr/JoorabchiMM14}. In addition, many studies have reported that the quality of issue reports impacts both the issue resolution time~\citep{DBLP:conf/cscw/BreuPSZ10}\citep{DBLP:conf/icse/GuoZNM10} and the issue resolution rate~\citep{DBLP:conf/compsac/ZouXZCL15}\citep{DBLP:conf/icse/ZimmermannNGM12}. 

To facilitate developers' bug-reproduction work, GitHub launched a new feature that allows users to share videos (e.g., mp4 files) in May 2021~\citep{github-video-blog}. Using such videos, reports can be made to developers about the details of bugs by recording the symptoms, reproduction steps, and other important aspects of a comprehensive bug report. These visual images can help developers understand the nature of the bug, and what users were doing when the bug occurred. While such visual issue reports have the potential to significantly improve the bug-fixing process, no studies have empirically examined this impact. 

In this paper, we conduct a preliminary study to identify the characteristics of visual issue reports by comparing them with non-visual issue reports.  In addition, we provide the dataset used in this study on \kashiwa{Where?}\footnote{URL here}, to promote future studies using visual issue reports. This dataset consists of videos and images in publicly available repositories on GitHub. Specifically, we collected 1,230 videos and 18,760 images from 226,286 issue reports on 4,173 publicly available repositories.


%Our initial analysis reveals that (i) visual issue reports still require reporters to write similar amounts of texts to describe bugs; (ii) visual issue reports are 3-times likely to receive more comments than reports without images or videos; and (iii) resolution time of visual issue reports is longer than that of other issue reports. 
Our initial analysis reveals that 
(i) issue reports with images are described in fewer words than non-visual issues; 
(ii) visual issue reports do not lead to active discussions in the number of words and the first response time; and 
(iii) resolution times of visual issue reports are not significantly changed compared to other issue reports. 
%The main contribution of this paper is opening a new research perspective to study visual issue reports and preparing a new dataset for this research. 