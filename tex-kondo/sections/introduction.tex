\section{Introduction}
\label{sec:intro}
{\it ``What makes good issue reports?''} has been studied for decades and 
is still an ultimate research question. 
Issue reports (a.k.a. bug reports) that report bugs with 
insufficient information delay fixing bugs~\citep{DBLP:conf/msr/JoorabchiMM14}. 
For example, Zimmermann~\et~\citep{zimmermann2010TSE} reported 
that the steps to reproduce bugs and stack traces are important 
to developers while such ones are occasionally missing or incorrect. 
This mismatch between what developers need and 
what reporters who write issue reports provide would seriously delay fixing bugs. 
Also, prior studies reported the quality of issue reports affects 
the issue resolution time\masa{need citation} and 
issue resolution rate\masa{need citation}. 


%% In recent years, distributed development~\citep{sengupta2006ICSE} 
%% such as remote work has become more important 
%% than in the last decades~\citep{ford2021TOSEM}. 
%% Indeed, developers in industry also employ remote work
%% in recent years~\citep{gitlab2020survey}.
%% % for work-life balance. 
%% To make distributed development successful, 
%% developers have to be aware of their communication 
%% in their communities. 
%% % For example, developers for open source software develop software
%% % at different times and in different regions, while
%% For example, developers for open source software are 
%% distributed around the world and live in different time zones, 
%% while developers in the same company generally gather around 
%% the office and live in the same time zone. 
%% GitLab Inc. in which all developers work remotely 
%% public their knowledge of the key steps 
%% to make remote work successful~\citep{gitlab2020remoteplaybook}. 

%% However, in general, distributed development still has 
%% the challenge in communication~\citep{ford2021TOSEM}. 
%% % \masa{ford's work is a good reference}
%% For example, in non-distributed development, 
%% developers would discuss a bug by showing the display of 
%% their laptop face-to-face. 
%% The questioner would easily share the way to reproduce the bug 
%% and the questionee could understand the details of the bug. 
%% On the contrary, in distributed development such as 
%% open source software, they have to discuss the bug 
%% in asynchronous communication 
%% such as text using issue tracking systems. 
%% In addition, it is generally difficult to explain the way 
%% to reproduce the bug in text communication 
%% even if it is easy to explain the bug by showing 
%% the display of their laptop. 
%% Hence, it would take a long time to fix the bug 
%% in distributed development. 


To address this challenge, GitHub released a new feature 
to easily share movies in May 2021~\citep{github-video-blog}. 
Developers and reporters can share their movies such as the mp4 format 
on GitHub (\eg,  issue reports, pull requests, and discussions). 
% Of course, they can use the images on GitHub as well. 
Visualized issue reports by this feature
% Such visualized issue reports 
(\textit{visual issue reports}) have the potential to 
easily share information without incorrect ones.  
For example, reporters can share the steps to reproduce bugs 
with the screen capture and developers can understand 
the steps quickly~\citep{github-video-blog}.
% even in asynchronous communication. 
% In addition, developers can share the background and 
% the importance of the complex changes easier with 
% only asynchronous communication. 
However, no studies about the impact of visual issue reports exist. 
Our ultimate goal is to establish the advantages of visual issue reports 
on software development and promote using them.


%% Asynchronous communication with 
%% %movies 
%% such visualization
%% has the potential 
%% to promote distributed development. 
%% Indeed, developers can use this feature to make communication 
%% in open source software development accelerate. 
%% However, no studies about asynchronous communication with 
%% the visualization in software development exist.
%% % as this feature is new. 
%% % Hence, it is important to prepare the dataset of the visualization
%% % for software development. 
%% Hence, it is important to prepare the dataset of the visualization 
%% and study it to reveal the impact of the visualization.



% We present a dataset. 
%% In this paper, we present the first dataset to accelerate 
%% the research about the impact of visualization 
%% in communication in GitHub and the initial analysis. 
As a precursor to this goal, in this paper, 
we conducted a preliminary study about the characteristics of 
visual issue reports in software development on GitHub
by comparing them to the issue reports with images and 
the other issue reports. 
In addition, to promote the research about visual issue reports, 
we built a dataset of visual issue reports. 
This dataset consists of movies and images 
in publicly available repositories on GitHub. 
Specifically, we collected 
3,819 movies and 
33,079 images from 
711,160 issue reports on
4,173 publicly available repositories.

Our initial analysis reveals that 
1) the visual issue reports are still not popular,
while the number of them has slightly increased in recent years, 
% 2) the issue reports with either images or movies need 
2) the visual issue reports need 
a longer time to be resolved than the issue reports without images and movies, and 
% 3) the issue reports with either images or movies are more related to 
3) the visual issue reports are more related to 
the topics of visualization and GUI. 
The main contribution of this paper is 
opening a new research perspective to study visual issue reports
% in asynchronous communication 
and preparing a new dataset for this research. 