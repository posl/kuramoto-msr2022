% \section{Limitation}
\subsection{Threats to Validity}
\label{sec:limitation}

As described in \sec{sec:design:context}, we randomly selected 
repositories because of the limitation of execution time. 
The proportion of studied repositories is less than 2\%. 
However, the number of resolved issue reports that 
we collected is 770,656. 
Hence, the number of resolved issue reports is enough. 

GitHub has released the feature in May 2021. 
Hence, this feature is still new. 
\fig{fig:data-cat-trend} shows that the number of 
issue reports that have videos has slightly increased so far.
% therefore, we can extend the dataset in the future. 
% This extension may change the characteristics of 
% the results based on the current dataset. 
We need to revise the results when this feature becomes more popular. 

We collected the issue reports with PyGitHub. 
However, in this process, some pull requests 
are also collected due to the specification of GitHub. 
Hence, our analysis may include such pull requests as issue reports. 