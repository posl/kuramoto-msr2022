\section{Related Work}
\label{sec:relate}
While numerous studies have worked on bug resolution time~\citep{DBLP:conf/msr/ChenNSH14}\citep{DBLP:journals/jss/GarciaSN18}\citep{DBLP:conf/sigsoft/JeongKZ09}\\
\citep{DBLP:conf/icsm/KashiwaYKO14}\citep{DBLP:conf/msr/ZamanAH11}, 
in particular, the following studies investigated the relationship between bug resolution time and various elements of bug reports other than videos~\citep{DBLP:conf/msr/BhattacharyaN11}\citep{DBLP:conf/csmr/BhattacharyaUNK13}\citep{DBLP:journals/ieicetd/NoyoriWFKONT21}. 
Noyori~\et~\citep{DBLP:journals/ieicetd/NoyoriWFKONT21} investigated the relationship between resolution time and topics included in the comments of issue reports. 
They found that bugs are resolved fast when discussions about symptoms are not needed.  Bhattacharya \et \citep{DBLP:conf/msr/BhattacharyaN11} developed bug-fix
time prediction models using various metrics. They showed that bug severity and the number of attachments (patches) do not correlate with bug-fix time. In addition, their later work by Bhattacharya \et~\citep{DBLP:conf/csmr/BhattacharyaUNK13} compared bug-fix time for high-quality and poor-quality reports. They observed that the text length of descriptions is relatively correlated with bug resolution time. 

A few recent studies have utilized visual issue reports for improving bug-fixing process~\citep{DBLP:conf/icse/CooperBCMP21}. Cooper~\et~\citep{DBLP:conf/icse/CooperBCMP21} used videos and texts included in issue reports to detect duplicate issue reports. Compared with the study, the contribution of our work is (1) the analysis of the impact of visual issue reports and (2) the public available datasets including 1,324 videos and 21,003 images from 241,398 issue reports.