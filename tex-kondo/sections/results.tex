\section{Results}
\label{sec:results}

\begin{table}[t]
  \begin{center}
  \caption{The Steel-Dwass test results \masa{finally we should remove this table}}
  \begin{tabular}{l r|r}
    \toprule
    Attributes & Category & $p$-value\\
    \midrule
     & \bf{$Img$~vs~$None$} & 0.535 \\
     $IssueResolvedTime$ & \bf{$Mov$~vs~$None$} & 0.636\\
     & \bf{$Img$~vs~$~Mov$} & 0.587 \\
    \midrule
     & \bf{$Img$~vs~$None$} & 0.491 \\
     $FirstCommentTime$ & \bf{$Mov$~vs~$None$} & 0.494 \\
     & \bf{$Img$~vs~$~Mov$} & 0.274 \\
    \midrule
     & \bf{$Img$~vs~$None$} & 0.216 \\
     $\#comments$ & \bf{$Mov$~vs~$None$} & 0.245 \\
     & \bf{$Img$~vs~$~Mov$} & 0.749 \\
    \midrule
     & \bf{$Img$~vs~$None$} & *~~0.002 \\
     $\#words$  & \bf{$Mov$~vs~$None$} &  0.435 \\
     & \bf{$Img$~vs~$~Mov$} & *~~0.012 \\
    \bottomrule
  \end{tabular}\\
  %\small
  % *~ : significant in two-sided test \\
  % ** : significant in one-sided test \\
  \label{tab:Steel-Dwass-test}
  \end{center}
\end{table}


% 初期段階の調査として,画像及び動画がissueの各attributesに
% 与える影響について調査する.
% 表2で得られたそれぞれのカテゴリ間で比較を比較を行う.
% 等分散性を仮定できないデータが含まれていたため,
% 比較にはSteel-Dwass testを採用する.
% 
% visualizationは,issueの解決時間,
% コメント数,文字数に何らかの影響を与える.
% 検定結果を表6に示す.
% 有意水準は0.05を採用する.
% IssueResolvedTimeにおいて,
% ImgとMovは片側検定で有意差があった.
% また,表Xより,平均値,中央値でNoneカテゴリの
% issueよりImgとMovカテゴリのissueの方が
% 課題解決時間が長かった.
% よって,画像もしくは動画がissue作成時に付与
% される課題はその解決時間が有意に長くなる.
% また,#comments及び#charsでは両側検定で
% 有意であり,これらのattributesにも
% 有意な影響を与えることがわかった.
% ただし,imageとmovieに有意差は
% 観測されなかった.

% As an initial analysis, we investigated 
% the impact of images and movies on 
% the issues in terms of the attributes. 
As an initial analysis, we investigated 
the differences with and without images and movies 
for each attribute on the issues. 
Specifically, we compared the attributes between 
the categories in \tab{tab:issue-category}. 
Because our preliminary study shows that 
the distributions for each category do not 
come from normal distributions, 
we used a non-parametric test called the \textit{Steel-Dwass test}. 

\textbf{The attribute values are changed 
with and without images and movies.}
\tab{tab:Steel-Dwass-test} shows the results of 
the Steel-Dwass test. 
The asterisks indicate significance based on 
the Steel-Dwass test: * indicates $p$ < 0.05 in 
the two-sided test; 
** indicates $p$ < 0.05 in the one-sided test. 
We observed that the issues in the $Img$ category and 
the $Mov$ category are significantly different from those 
in the $None$ category in terms of $IssueResolvedTime$. 
Also, \tab{tab:issue_stat_categories} shows that 
the mean and the median $IssueResolvedTime$ values of 
the issues in the $Img$ category and the $Mov$ category are 
longer than those in the $None$ category. 
Hence, the issues with images or movies tend to need 
a longer time to be resolved. 

In addition, we observed that the issues in 
the $Img$ category and the $Mov$ category are 
significantly different from those in 
the $None$ category in terms of $\#comments$ and 
$\#chars$. 
Hence, the issues with images or movies 
tend to have different numbers of 
comments and words in the issue description.

It should be noted that we do not observe 
the difference between the $Img$ category and 
the $Mov$ category. 