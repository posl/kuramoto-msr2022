\section{RQの回答と考察\label{research}}

本節では,5節の調査結果からRQ1~RQ3について回答し,サンプルサイズについて考察する.

5.1節の結果から,動画及び画像の活用頻度として以下の結果を得た.
\begin{itembox}{\textbf{RQ1:動画及び画像の活用頻度はどの程度か.}}
  ・2016年以前はほとんど用いられてこなかった.\\
  ・2017年に急激に増加し,以降増加傾向である.\\
  ・2021年には,動画0.72\%,画像5.93\%であった.
\end{itembox}

5.2節では,動画及び画像の有無で,Issueのパラメータがどのように変化するかを分析した.
また,5.3節では,各調査項目の群間の分布の差が統計的に有意であるかを確かめた.
動画及び画像を含むIssueは,そうでないIssueに比べて,
$Num\_of\_char$が平均で26.3\%\textasciitilde 37.5\%減少したことから,
動画及び画像を用いることでバグの症状を細かく記述せずに済むと推察される.
また,$First\_comment\_time$が平均で20.1\%\textasciitilde25.4\%減少し,
$Num\_of\_comments$が平均で76.4\%\textasciitilde143\%増加したことから,
動画及び画像のあるIssueは開発者の着手を早め,コメントを返信が得やすくなると推察される.
また,$Img$と$Mov$の比較検定では,すべての項目で有意差が認められなっかことから,
動画と画像の効果に有意な差はないと推察される.
%これらの結果から,動画及び画像の期待される効果として,以下が挙げられる.
\begin{itembox}{\textbf{RQ2:動画及び画像の期待される効果は何か.}}
  ・バグの症状を細かく記述せずに済む.\\
  ・開発者の着手を早め,コメントを得やすい.\\
  ・動画と画像の効果に有意な差はない.
\end{itembox}

5.4節では,出現単語分析を行うことで,どの様な報告内容の場合に動画及び画像を用いるかを確かめた.
$Img$では表示に関係する単語が多く,$Mov$では動きのある表示に関する単語が多く見られた.
一方で$None$では,プルリクエストやコミットなどリポジトリ管理に関する単語が多かった.

\begin{itembox}{\textbf{RQ3:どの様な報告に動画及び画像を用いるか.}}
  ・表示に関する問題に対して,画像を用いる.\\
  ・動きのある表示に関する問題に対して,動画を用いる.
\end{itembox}

一般にサンプルサイズは多ければ多いほど,母集団の特徴をより正確に抽出できる.
ところが検定においては,サンプルサイズが大きくなるにつれて,誤差に過敏になり有意確率が小さくなる傾向がある.
すなわち,検定で有意差を得たとしても,母集団分布の有意な差によるものか,サンプルサイズが大きすぎることによるものか判断できない.
従って,適切なサンプルサイズを前もって決めておくことが重要である.
本調査では,事前に効果量の目安が得られなかったために,やや恣意的に決定した.
検定結果を見ると,有意差がはっきりと認められた部分と,認められなかった部分が見られた$Issue\_open\_time$,$Num\_of\_comments$及び$Num\_of\_char$に対しては適切なサンプルサイズであったと推察する.
一方で$First\_comment\_time$に対しては,実際に有意と言える差が無いか,あるいは,サンプルサイズが小さすぎた可能性がある.

%また本調査では,すべての調査項目が正規分布に従わなかった.
%一般に,正規分布を仮定しない検定法よりも,正規分布を仮定する検定法は検出力に長けていると言われている.
%従って,データが正規分布に従わない場合,正規分布に近似するため適切な変数変換を行う.
%時系列データの変換には対数変換が多くの場合有効である.本調査の調査項目の一部は時系列データであり,
%変換が推奨される場面であるが,実際に適用させたところ平均値の群間順序関係が変化したことと,
%順位和検定により順位に変換されることから,対数変換を用いずに調査した.