\section{動機\label{motivation}}

GitHubでは,公開されたリポジトリに対して,そのオーナーだけでなく不特定多数のユーザーが開発に貢献しており,
不具合や意見の報告時に Issue機能が使用されている.
2021年以前は,Issue作成には文字入力の他に,GIF動画や画像のみ添付可能であり,
動画添付時には一度GIF動画に変換する手間があった.
現在では,MP4,MOVファイルを容易に添付できるようになり,今後動画の活用件数は増加すると考えられる.
しかし現時点では,動画及び画像の有無によるIssueへの影響について明らかにされていない.
本調査では,次の3つのRQを設定し,回答する.

\textbf{RQ1:動画及び画像の活用頻度はどの程度か.}\\
RQ1では,動画及び画像のどの程度の頻度で用いられているかを明らかにする.

\textbf{RQ2:動画及び画像の期待される効果は何か.}\\
RQ2では,動画及び画像の有無が,解決までの時間やコメントの回数などIssueのパラメータにどのように影響するかを明らかにする.

\textbf{RQ3:どのような報告内容に動画及び画像を用いるか.}\\
RQ3では,動画及び画像がどのような単語と併用されているかを調査することで,
どのような報告内容のときに動画及び画像が用いられているかを明らかにする.